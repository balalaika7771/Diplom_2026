\chapter{Практическая значимость}

\section{Применение в индустрии}

Разработанная система применима в следующих классах распределённых инфраструктур \cite{amazon_outage_cost,aiops_survey}:

\begin{enumerate}
\item \textbf{Облачные провайдеры} — автоматизация операций в крупных облачных платформах (AWS, Azure, GCP), снижение операционных затрат и риска SLA-штрафов.
\item \textbf{Финансовые системы} — обеспечение высокой доступности критически важных систем, уменьшение прямых финансовых потерь и репутационного ущерба.
\item \textbf{Электронная коммерция} — минимизация простоя торговых платформ, сокращение потерь выручки и отказов клиентов.
\item \textbf{Телекоммуникации} — автоматическое восстановление сетевых сервисов, контроль выполнения SLA по задержкам и доступности.
\item \textbf{Промышленная автоматизация} — мониторинг и восстановление производственных систем, снижение вероятности остановки производственных линий.
\end{enumerate}

\section{Интеграция в реальные системы}

Интеграция возможна за счёт опоры на стандартные инструменты наблюдаемости и оркестрации:

\begin{enumerate}
    \item \textbf{Prometheus} — интеграция через exporters и alertmanager \cite{prometheus_docs}
    \item \textbf{Kubernetes} — использование операторов и CRD \cite{burns_kubernetes,kubernetes_docs}
    \item \textbf{Grafana} — плагины для визуализации \cite{grafana_docs}
    \item \textbf{OpenTelemetry} — сбор данных через стандартизированные протоколы \cite{opentelemetry,opentelemetry_docs}
\end{enumerate}

Требования к интеграции:
\begin{itemize}
    \item API для взаимодействия с существующими системами
    \item Поддержка стандартных протоколов (REST, gRPC) \cite{rfc_7540,grpc_design}
    \item Совместимость с существующими форматами данных
    \item Минимальные изменения в существующей инфраструктуре
\end{itemize}

\section{Масштабируемость решения}

Система спроектирована для масштабирования по числу сервисов и агентов. Для малых кластеров (десятки сервисов) достаточно одного–трёх агентов и стандартных настроек хранилищ наблюдаемости. Для средних и крупных кластеров (сотни и тысячи сервисов) предусмотрено горизонтальное масштабирование агентов, шардирование хранилищ и поэтапное развертывание политик восстановления, что позволяет сохранять приемлемое время реакции при росте нагрузки.

\section{Экономическая оценка эффекта}

Для иллюстрации практического эффекта рассмотрим типовой сценарий отказа в системе электронной коммерции. Пусть средняя стоимость одной минуты простоя критического сервиса равна \(C_{\text{min}}\) (например, \$50\,000 в минуту \cite{amazon_outage_cost}), базовое среднее время восстановления \(\text{MTTR}_{\text{baseline}}\), а достигнутое в результате внедрения системы агентов среднее время восстановления \(\text{MTTR}_{\text{agent}}\) заданы согласно главе~11.

Тогда ожидаемое снижение прямых потерь от простоя за один инцидент оценивается как
\begin{equation}
  \Delta \text{Loss} = C_{\text{min}} \cdot \left(\text{MTTR}_{\text{baseline}} - \text{MTTR}_{\text{agent}}\right),
  \label{eq:economic_loss_delta}
\end{equation}
а годовой эффект при среднем числе инцидентов \(N_{\text{inc}}\) в год:
\begin{equation}
  \Delta \text{Loss}_{\text{year}} = N_{\text{inc}} \cdot \Delta \text{Loss}.
  \label{eq:economic_loss_year}
\end{equation}

Подставляя, например, усреднённые значения из таблицы~\ref{tab:experiments_summary} (глава~11): \(\text{MTTR}_{\text{baseline}} \approx 78.75\) мин, \(\text{MTTR}_{\text{agent}} \approx 44\) мин, \(C_{\text{min}} = \$50\,000\), получаем
\[
  \Delta \text{Loss} \approx 50\,000 \cdot (78.75 - 44) \approx 1.7375 \cdot 10^9 \ \text{USD}
\]
экономии на один крупный инцидент такого класса. Даже при более консервативных предположениях о стоимости простоя и частоте инцидентов выражение~(\ref{eq:economic_loss_year}) показывает, что снижение \(\text{MTTR}\) на десятки процентов приводит к эффекту, существенно превышающему стоимость внедрения системы агентов.

Содержательно данная глава связывает разработанный теоретический и экспериментальный аппарат с конкретными классами промышленных систем, в которых автоматизация диагностики и самовосстановления даёт наибольший эффект. Через оценку потенциальной экономии и снижения операционных рисков для облачных платформ, финансовых систем, e-commerce и телеком-инфраструктур показывается, какие типы инцидентов и эксплуатационных ограничений прежде всего адресуются предлагаемым решением.

Модульная архитектура и формализованная унификация данных обеспечивают совместимость с распространёнными стеками наблюдаемости и оркестрации, а анализ масштабируемости демонстрирует возможность применения подхода в диапазоне от малых до крупномасштабных кластеров. Таким образом, практическая значимость работы выражается в наличии чётко очерченного класса реальных систем, для которых внедрение разработанной системы интеллектуальных агентов приводит к измеримому улучшению показателей надёжности и операционной эффективности.

