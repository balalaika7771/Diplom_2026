\chapter{Практическая значимость и экономическая эффективность}

\section{Области применения}

Разработанная система применима в распределённых инфраструктурах, удовлетворяющих двум условиям: (1)~наличие стека наблюдаемости с метриками, логами и трейсами; (2)~критичность времени восстановления для бизнес-процессов. Рассмотрим конкретные классы систем \cite{amazon_outage_cost,aiops_survey}:

\begin{table}[H]
\centering
\caption{Целевые области применения и ожидаемый эффект}
\label{tab:application_areas}
\small
\begin{tabular}{|p{2.8cm}|p{3.5cm}|p{2.5cm}|p{3cm}|}
\hline
\textbf{Область} & \textbf{Типовые инциденты} & \textbf{Стоимость простоя} & \textbf{Ожидаемый эффект} \\
\hline
Облачные платформы & Отказ VM, деградация сети, перегрузка API & \$5--50K/мин & Снижение MTTR на 40--45\% \\
\hline
Финансовые системы & Задержки транзакций, сбои процессинга & \$10--100K/мин & Рост Availability до 99.95\% \\
\hline
E-commerce & Недоступность каталога, сбои оплаты & \$1--20K/мин & Снижение потерь выручки \\
\hline
Телеком & Деградация качества, отказы узлов & \$2--30K/мин & Автоматизация SLA-контроля \\
\hline
\end{tabular}
\end{table}

Стоимость простоя приведена по данным Gartner \cite{amazon_outage_cost} и варьируется в зависимости от масштаба и критичности системы.

\section{Интеграция в существующую инфраструктуру}

Модульная архитектура системы обеспечивает интеграцию через стандартные интерфейсы:

\begin{enumerate}
    \item \textbf{Сбор данных}: OpenTelemetry Collector для унификации метрик, логов и трейсов \cite{opentelemetry,opentelemetry_docs}. Поддержка Prometheus remote\_write, Elasticsearch bulk API, Jaeger gRPC.
    \item \textbf{Оркестрация}: Kubernetes Operator с Custom Resource Definitions (CRD) для декларативного управления агентами \cite{burns_kubernetes,kubernetes_docs}.
    \item \textbf{Действия восстановления}: REST/gRPC API для интеграции с CI/CD (ArgoCD, Flux) и runbook-системами (PagerDuty, Opsgenie) \cite{rfc_7540,grpc_design}.
    \item \textbf{Визуализация}: Grafana-плагин для отображения состояния агентов, причинных графов и истории действий \cite{grafana_docs}.
\end{enumerate}

Минимальные требования к инфраструктуре для внедрения: Kubernetes 1.25+, Prometheus с retention $\geq$ 7 дней, любая система агрегации логов с поддержкой structured logging.

\section{Масштабируемость}

Архитектура масштабируется горизонтально. Зависимость потребления ресурсов от числа наблюдаемых сервисов $n$:

\begin{equation}
\text{CPU}_{\text{agent}} = O(n \log n), \quad \text{RAM}_{\text{agent}} = O(n \cdot w)
\label{eq:scalability}
\end{equation}

где $w$ --- размер окна анализа (количество точек данных). Логарифмический фактор в CPU обусловлен алгоритмом построения причинного графа (сортировка по времени и поиск зависимостей).

\begin{table}[H]
\centering
\caption{Масштабируемость: потребление ресурсов при росте числа сервисов}
\label{tab:scalability}
\small
\begin{tabular}{|c|c|c|c|c|}
\hline
\textbf{Сервисов} & \textbf{Агентов} & \textbf{CPU, cores} & \textbf{RAM, ГБ} & \textbf{Latency диагн., с} \\
\hline
10 & 1 & 0.5 & 1.4 & 3.2 \\
\hline
50 & 2 & 1.8 & 4.1 & 5.8 \\
\hline
100 & 3 & 3.5 & 7.3 & 8.4 \\
\hline
500 & 8 & 12.1 & 24.6 & 14.2 \\
\hline
1000 & 15 & 28.4 & 51.2 & 22.7 \\
\hline
\end{tabular}
\end{table}

\section{Экономическая оценка}

\subsection{Модель расчёта экономического эффекта}

Годовой экономический эффект от внедрения системы определяется разницей между снижением потерь от простоев и стоимостью владения:

\begin{equation}
\text{NPV}_{\text{year}} = \underbrace{N_{\text{inc}} \cdot C_{\min} \cdot (\text{MTTR}_b - \text{MTTR}_a)}_{\text{Экономия от снижения простоев}} - \underbrace{C_{\text{infra}} + C_{\text{ops}} + C_{\text{dev}}}_{\text{Стоимость владения (TCO)}}
\label{eq:economic_npv}
\end{equation}

где $N_{\text{inc}}$ --- среднее количество инцидентов в год, $C_{\min}$ --- стоимость минуты простоя, $\text{MTTR}_b$ и $\text{MTTR}_a$ --- среднее время восстановления до и после внедрения, $C_{\text{infra}}$ --- стоимость инфраструктуры для агентов, $C_{\text{ops}}$ --- операционные расходы на поддержку, $C_{\text{dev}}$ --- стоимость начальной разработки и настройки (амортизируется).

\subsection{Расчёт для типового сценария}

Рассмотрим компанию среднего масштаба с 50 микросервисами:

\begin{table}[H]
\centering
\caption{Параметры расчёта экономического эффекта}
\label{tab:economic_params}
\small
\begin{tabular}{|p{5cm}|c|p{4cm}|}
\hline
\textbf{Параметр} & \textbf{Значение} & \textbf{Источник} \\
\hline
Инцидентов в год, $N_{\text{inc}}$ & 240 & 20 инцидентов/мес \\
\hline
Стоимость простоя, $C_{\min}$ & \$500/мин & Медиана для среднего SaaS \cite{amazon_outage_cost} \\
\hline
$\text{MTTR}_b$ & 79.1 мин & Таблица~\ref{tab:experiments_summary} \\
\hline
$\text{MTTR}_a$ & 43.7 мин & Таблица~\ref{tab:experiments_summary} \\
\hline
Инфраструктура агентов, $C_{\text{infra}}$ & \$18\,000/год & 2 сервера + облачные ресурсы \\
\hline
Операционные расходы, $C_{\text{ops}}$ & \$12\,000/год & 0.1 FTE инженера \\
\hline
Разработка/настройка, $C_{\text{dev}}$ & \$15\,000 & Единоразово, амортизация 3 года \\
\hline
\end{tabular}
\end{table}

Подставляя значения для базового сценария ($C_{\min} = \$500$/мин):

\begin{align}
\text{Экономия} &= 240 \cdot 500 \cdot (79.1 - 43.7) = 240 \cdot 500 \cdot 35.4 \notag \\
&= 4\,248\,000 \ \text{USD/год} \label{eq:savings_calc}
\end{align}

\begin{equation}
\text{TCO}_{\text{год}} = 18\,000 + 12\,000 + \frac{15\,000}{3} = 35\,000 \ \text{USD/год}
\label{eq:tco_calc}
\end{equation}

\begin{equation}
\text{ROI} = \frac{\text{Экономия} - \text{TCO}}{\text{TCO}} = \frac{4\,248\,000 - 35\,000}{35\,000} \approx 120
\label{eq:roi}
\end{equation}

Срок окупаемости:
\begin{equation}
T_{\text{payback}} = \frac{C_{\text{dev}} + C_{\text{infra}}}{\text{Экономия}_{\text{мес}}} = \frac{15\,000 + 18\,000}{4\,248\,000/12} \approx 0.093 \ \text{года} \approx 1.1 \ \text{мес}
\label{eq:payback}
\end{equation}

Стоимость простоя существенно варьируется в зависимости от масштаба бизнеса. Для корректной оценки рассмотрим три сценария:

\begin{table}[H]
\centering
\caption{Экономическая эффективность при различной стоимости простоя}
\label{tab:roi_scenarios}
\small
\begin{tabular}{|p{3cm}|c|c|c|c|}
\hline
\textbf{Сценарий} & \textbf{$C_{\min}$, USD} & \textbf{Экономия/год} & \textbf{ROI} & \textbf{Окупаемость} \\
\hline
Консервативный & 100 & \$849\,600 & 23 & 5.4 мес \\
\hline
\textbf{Базовый} & \textbf{500} & \textbf{\$4\,248\,000} & \textbf{120} & \textbf{1.1 мес} \\
\hline
Оптимистичный & 5\,000 & \$42\,480\,000 & 1\,213 & 3 дня \\
\hline
\end{tabular}
\end{table}

Консервативный сценарий (\$100/мин) соответствует небольшим внутренним системам, базовый (\$500/мин) --- среднему SaaS-бизнесу, оптимистичный (\$5\,000/мин) --- крупным платформам уровня Tier-1 \cite{amazon_outage_cost}. Экономическая целесообразность подтверждается даже в консервативном сценарии: ROI = 23 при окупаемости менее 6 месяцев.

\subsection{Анализ чувствительности}

Экономический эффект зависит от трёх ключевых параметров: стоимости простоя, частоты инцидентов и достигнутого снижения MTTR. На рисунке~\ref{fig:sensitivity} показана зависимость годовой экономии от стоимости простоя при различной частоте инцидентов.

\begin{figure}[H]
\centering
\begin{tikzpicture}[scale=0.9]
\begin{axis}[
    xlabel={Стоимость простоя, USD/мин},
    ylabel={Годовая экономия, млн USD},
    title={Анализ чувствительности экономического эффекта},
    grid=major,
    legend pos=north west,
    xmin=100, xmax=10000,
    ymin=0,
    xmode=log,
]
\addplot[blue,thick,mark=square] coordinates {
    (100, 0.85) (500, 4.25) (1000, 8.5) (5000, 42.5) (10000, 85.0)
};
\addplot[red,thick,mark=triangle] coordinates {
    (100, 0.42) (500, 2.12) (1000, 4.25) (5000, 21.2) (10000, 42.5)
};
\addplot[green!60!black,thick,mark=*] coordinates {
    (100, 0.17) (500, 0.85) (1000, 1.70) (5000, 8.5) (10000, 17.0)
};
\legend{240 инц./год, 120 инц./год, 48 инц./год}
\end{axis}
\end{tikzpicture}
\caption{Анализ чувствительности: зависимость экономии от стоимости простоя}
\label{fig:sensitivity}
\end{figure}

Экономическая целесообразность сохраняется при стоимости простоя от \$100/мин и частоте инцидентов от 4 в месяц, что соответствует большинству production-систем масштаба более 20 сервисов.
