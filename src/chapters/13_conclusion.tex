\chapter{Заключение}

\section{Основные результаты работы}

В ходе выполнения работы решены поставленные задачи и достигнута цель исследования. Основные результаты:

\begin{enumerate}
    \item Проведён комплексный анализ теоретических основ распределённых вычислительных систем, концепции наблюдаемости, backend-подходов к устойчивости, инженерных парадигм DevOps и SRE, методов обнаружения аномалий и архитектур автономных агентов.
    
    \item Разработана архитектура интеллектуальных автономных агентов для диагностики и самовосстановления распределённых систем, включающая модули сбора данных, анализа аномалий, диагностики, оркестрации и безопасного исполнения.
    
    \item Реализован прототип системы, демонстрирующий работоспособность предложенной архитектуры и готовность к практическому применению.
    
    \item Проведено экспериментальное исследование, показавшее снижение MTTR на 44-45\%, повышение точности обнаружения аномалий до 94\%, улучшение F1-Score с 0.68 до 0.90.
    
    \item Оценена практическая значимость решения для различных отраслей индустрии.
\end{enumerate}

\begin{table}[H]
\centering
\caption{Сводка достигнутых результатов}
\label{tab:results_summary}
\begin{tabular}{|l|c|c|}
\hline
\textbf{Метрика} & \textbf{Целевое значение} & \textbf{Достигнутое значение} \\
\hline
MTTR снижение, \% & 40 & 44-45 \\
\hline
Точность обнаружения, \% & 90 & 94 \\
\hline
F1-Score & 0.85 & 0.90 \\
\hline
Ложные срабатывания, \% & < 15 & 8.5 \\
\hline
\end{tabular}
\end{table}

\section{Научная новизна и вклад}

Научный вклад работы включает:

\begin{enumerate}
    \item архитектуру интеллектуальных автономных агентов, объединяющую методы машинного обучения, причинно-следственный анализ и механизмы безопасного самовосстановления;
    \item подход к построению причинно-следственных графов инцидентов на основе совместного анализа метрик, логов и трейсов;
    \item методологию интеграции больших языковых моделей для анализа логов и генерации гипотез о причинах инцидентов;
    \item систему метрик для оценки эффективности агентов, учитывающую точность обнаружения, безопасность и эффект действий по восстановлению.
\end{enumerate}

\section{Практическая значимость}

Практическая значимость работы состоит в том, что:

\begin{enumerate}
    \item разработанные методы применимы в реальных распределённых системах для повышения надёжности и доступности;
    \item автоматизация диагностики и восстановления снижает операционные затраты и время простоя;
    \item архитектура совместима с распространёнными платформами наблюдаемости и оркестрации;
    \item решение пригодно для облачных платформ, финансовых систем, систем электронной коммерции и телекоммуникаций.
\end{enumerate}

\section{Перспективы развития}

Перспективные направления дальнейших исследований:

\begin{enumerate}
    \item Расширение возможностей LLM для более глубокого анализа логов и автоматической генерации решений по восстановлению.
    
    \item Разработка методов обучения с подкреплением для оптимизации стратегий восстановления \cite{goodfellow_deep_learning,murphy_machine_learning}.
    
    \item Интеграция с системами прогнозирования нагрузки для превентивного масштабирования.
    
    \item Разработка многоагентных систем с координацией между агентами для сложных распределённых систем \cite{wooldridge_agents,multi_agent_systems,autonomous_agents_phd}.
    
    \item Создание стандартов и протоколов для взаимодействия интеллектуальных агентов различных производителей.
\end{enumerate}

\section{Общие выводы}

Работа посвящена проектированию интеллектуальных автономных агентов для диагностики, мониторинга и самовосстановления распределённых вычислительных систем на основе данных наблюдаемости. Гипотеза (формула~(\ref{eq:hypothesis_ineq})) о снижении среднего времени восстановления \(MTTR\) и росте интегральной надёжности \(R\) при использовании таких агентов проверяется с опорой на формальные модели (главы 2–8), архитектуру и прототип (главы 8–10) и экспериментальное исследование (глава 11).

Разработанная архитектура и методы снижают время восстановления после инцидентов, повышают точность обнаружения аномалий и обеспечивают безопасное автоматическое восстановление систем. В терминах метрик \(MTTR\), \(MTBF\) и доступности \(A\) (формулы~(\ref{eq:mttr}), (\ref{eq:availability})) это соответствует переходу системы в область большей доступности при сопоставимом уровне ресурсоёмкости.

Результаты подтверждены экспериментально: достигнутые значения снижения \(MTTR\), повышения точности и уменьшения доли ложных срабатываний сопоставимы или превосходят целевые ориентиры, заданные во введении и суммарно представленные в таблице~\ref{tab:results_summary}.

Предложенное решение может быть успешно применено в различных отраслях индустрии для повышения надёжности и снижения операционных затрат распределённых вычислительных систем.

Сводя результаты, получаем: комплексная интеграция наблюдаемости, методов обнаружения аномалий, причинно-следственного анализа и архитектур интеллектуальных агентов позволяет построить практическую систему автоматической диагностики и самовосстановления распределённых вычислительных систем с измеримым улучшением эксплуатационных характеристик. Разработанные математические конструкции — от моделей надёжности и временных рядов до причинно-следственных графов и алгоритмов консенсуса — образуют согласованный аппарат, реализованный в архитектуре и прототипе системы и подтверждённый экспериментально.

Итоговый экспериментальный анализ показывает, что при соблюдении заданных ограничений по безопасности и риску возможно существенно снизить MTTR и повысить точность обнаружения инцидентов, не увеличивая при этом нагрузку на эксплуатационный персонал. Это подтверждает исходную гипотезу о том, что интеллектуальные автономные агенты, опирающиеся на данные наблюдаемости и формальные модели принятия решений, способны существенно повысить надёжность распределённых систем.

С учётом проведённого анализа теоретических основ, архитектуры, прототипа и результатов экспериментов предложенный подход является состоятельным как с научной, так и с прикладной точки зрения и может служить основой для дальнейшего развития систем автоматизации операций в распределённых вычислительных средах.

