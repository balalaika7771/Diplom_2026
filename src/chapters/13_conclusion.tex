\chapter*{ЗАКЛЮЧЕНИЕ}
\addcontentsline{toc}{chapter}{ЗАКЛЮЧЕНИЕ}

В работе решена задача проектирования интеллектуальных автономных агентов для диагностики и самовосстановления распределённых вычислительных систем. Получены следующие основные результаты.

\section{Результаты работы}

\begin{enumerate}
    \item Проведён систематический анализ теоретических основ распределённых систем (теорема CAP, модели надёжности, каскадные отказы), концепции наблюдаемости (метрики, логи, трейсы), методов обнаружения аномалий (статистические, ML, LLM) и архитектур автономных агентов (реактивные, BDI, гибридные). Анализ выявил отсутствие комплексного решения, объединяющего обнаружение, диагностику и безопасное восстановление в единую архитектуру.

    \item Разработана модульная архитектура системы, включающая пять компонентов: Anomaly Detector (LSTM-сеть для обнаружения аномалий), Causal Diagnostic Engine (построение причинно-следственных графов), LLM Agent (анализ неструктурированных логов), Multi-Agent Consensus (согласование решений) и Safe Executor (валидация и безопасное исполнение). Проведён формальный анализ связности модулей ($\text{Coupling} = 0.24$) и обоснован выбор архитектурных парадигм.

    \item Реализован прототип системы на Java/Spring Boot с интеграцией в Kubernetes, Prometheus, Elasticsearch, Jaeger и Kafka. Реализован MCP-сервер для взаимодействия с LLM-моделями.

    \item Экспериментальное исследование на 4 сценариях инцидентов (отказ сервиса, каскадный отказ, утечка памяти, сетевая деградация) с 30 повторениями каждый подтвердило эффективность подхода. Ablation study установило иерархию значимости компонентов: LSTM Anomaly Detector ($\Delta F_1 = -17.9\%$) $>$ Causal Engine ($-10.1\%$) $>$ LLM Agent ($-4.6\%$) $>$ Consensus ($-2.9\%$).

    \item Апробация в тестовой среде (8 недель, 312 инцидентов) и пилотное внедрение в shadow-режиме (12 недель, 47 сервисов) подтвердили практическую применимость. Экономический анализ показал ROI $> 100$ при стоимости простоя от \$500/мин.
\end{enumerate}

\section{Достигнутые показатели}

\begin{table}[H]
\centering
\caption{Сводка достигнутых результатов}
\label{tab:results_summary}
\begin{tabular}{|l|c|c|c|}
\hline
\textbf{Метрика} & \textbf{Целевое} & \textbf{Достигнутое} & \textbf{$p$-value} \\
\hline
Снижение MTTR, \% & 40 & $44.8 \pm 3.2$ & $< 0.001$ \\
\hline
Precision, \% & 90 & $91.8 \pm 3.4$ & $< 0.001$ \\
\hline
$F_1$-Score & 0.85 & $0.905 \pm 0.02$ & $< 0.001$ \\
\hline
Ложные срабатывания, \% & $< 15$ & $8.2 \pm 2.8$ & --- \\
\hline
Availability & 99.0\% & 99.4\% & $< 0.01$ \\
\hline
\end{tabular}
\end{table}

Все целевые показатели достигнуты. Гипотеза $\text{MTTR}_{\text{agent}} < \text{MTTR}_{\text{baseline}}$, $R_{\text{agent}} > R_{\text{baseline}}$ подтверждена на уровне значимости $\alpha = 0.05$ по всем сценариям.

\section{Научный вклад}

\begin{enumerate}
    \item Комплексная архитектура интеллектуальных агентов, объединяющая ML-обнаружение аномалий, причинно-следственную диагностику и механизмы безопасного самовосстановления.
    \item Метод построения причинно-следственных графов инцидентов на основе совместного анализа метрик, логов и трейсов с формализацией через критерий Грейнджера.
    \item Методология интеграции LLM для анализа неструктурированных логов через MCP-протокол, повышающая скорость диагностики на 12.8\% (по ablation study).
    \item Механизм безопасного самовосстановления с формальной оценкой риска ($r_{\max} = 0.3$), заблокировавший 18 потенциально опасных действий за 20 недель апробации при нулевом уровне некорректных операций.
\end{enumerate}

\section{Ограничения и направления развития}

Выявленные ограничения определяют направления дальнейших исследований:

\begin{enumerate}
    \item \textbf{Ложные тревоги при деплоях}. 60\% ложных срабатываний связаны с плановыми изменениями инфраструктуры. Решение --- интеграция с CI/CD pipeline для автоматического подавления алертов в окнах деплоя.

    \item \textbf{Медленная деградация}. Инциденты с периодом развития $> 4$ часов пропускаются в 6\% случаев. Направление --- расширение временного окна LSTM и применение сезонных моделей (Prophet, DeepAR).

    \item \textbf{Многокластерные сценарии}. Текущая архитектура оптимизирована для одного кластера. Федеративный подход с координацией агентов между кластерами через gossip-протокол является предметом отдельного исследования.

    \item \textbf{Обучение с подкреплением}. Замена эвристических весов консенсуса ($\alpha_{ML}:\alpha_{LLM}:\alpha_{rule}$) на адаптивную стратегию, обучаемую по reward-сигналу от результатов восстановления \cite{goodfellow_deep_learning,murphy_machine_learning}.
\end{enumerate}

\section{Общие выводы}

Комплексная интеграция наблюдаемости, машинного обучения, причинно-следственного анализа и архитектур интеллектуальных агентов позволяет построить систему автоматической диагностики и самовосстановления распределённых систем с измеримым улучшением эксплуатационных характеристик. Экспериментальные результаты --- снижение MTTR на 44.8\%, рост $F_1$ до 0.905, снижение FPR до 8.2\% --- подтверждают исходную гипотезу на уровне статистической значимости $p < 0.001$.

\textbf{Значение для индустрии.} Результаты работы имеют прямую практическую значимость для SRE- и DevOps-команд, эксплуатирующих микросервисные архитектуры. Сокращение MTTR на 44.8\% при стоимости простоя от \$500/мин (ROI $> 100$, глава~12) позволяет обосновать внедрение подобных систем экономически. Модульная архитектура с формализованными интерфейсами между компонентами (Anomaly Detector, Diagnostic Engine, LLM Agent, Safe Executor) допускает поэтапное внедрение: организация может начать с Anomaly Detector в shadow-режиме и последовательно подключать остальные модули по мере накопления доверия к системе.

\textbf{Переносимость.} Архитектура не привязана к конкретному стеку. Прототип реализован на Java/Spring Boot с Prometheus, Elasticsearch и Jaeger, однако модульность позволяет заменить источники данных (Datadog вместо Prometheus, OpenSearch вместо Elasticsearch) при сохранении алгоритмической части. Единственное требование --- калибровка порогов LSTM-детектора и весов ансамбля на данных целевой среды, что подтверждено 20-недельной апробацией (глава~11).

Предложенный подход является состоятельным как с научной, так и с прикладной точки зрения и может служить основой для развития систем автоматизации операций в распределённых вычислительных средах.
