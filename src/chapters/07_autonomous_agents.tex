\chapter{Интеллектуальные автономные агенты и их архитектура}

\section{Введение: что такое автономный агент}

Автономный агент — программная система, способная действовать самостоятельно в заданной среде для достижения целей \cite{wooldridge_agents,multi_agent_systems}. В распределённых системах агент анализирует метрики, логи и трейсы, обнаруживает проблемы, определяет их причины и принимает решения о восстановлении без участия оператора \cite{autonomous_agents_phd}.

Традиционные системы мониторинга опираются на ручную диагностику и восстановление, что ограничивает скорость реакции и делает результаты зависящими от человеческого фактора.

Автономный агент характеризуется следующими свойствами:
\begin{itemize}
    \item Автономность — способность действовать без прямого вмешательства человека
    \item Реактивность — способность реагировать на изменения в среде в реальном времени
    \item Проактивность — способность инициировать действия для достижения целей, а не только реагировать на события
    \item Социальность — способность взаимодействовать с другими агентами для координации действий
\end{itemize}

В данной главе рассматриваются математические модели различных архитектур агентов и методов принятия решений.

\section{Проблема автоматизации принятия решений}

Необходимо заменить ручной выбор действий механизмом, который по наблюдаемому состоянию системы автоматически выбирает действие восстановления.

Математическая формулировка задачи: требуется функция принятия решений

\begin{equation}
  a^* = \arg\max_{a \in \mathcal{A}} U(s, a),
  \label{eq:decision_problem}
\end{equation}
где $\mathcal{A}$ — пространство действий, $s$ — текущее состояние системы, $U(s, a)$ — функция полезности действия $a$ в состоянии $s$.

\section{Реактивные архитектуры: решение проблемы быстрого реагирования}

Реактивные архитектуры основаны на обработке событий без построения сложных моделей среды \cite{agent_architectures}. Проблема традиционных подходов — задержка в реакции на изменения.

Реактивный агент функционирует по циклу восприятие-решение-действие, который формализуется следующим образом:

\begin{equation}
\text{Agent}(s_t) = \text{Act}(\text{Decide}(\text{Perceive}(s_t)))
\label{eq:reactive_agent}
\end{equation}

где $\text{Perceive}: \mathcal{S} \to \mathcal{O}$ — функция восприятия, $\text{Decide}: \mathcal{O} \to \mathcal{A}$ — функция принятия решений, $\text{Act}: \mathcal{A} \to \mathcal{S}$ — функция действия.

Время реакции реактивного агента:

\begin{equation}
T_{\text{reaction}} = T_{\text{perceive}} + T_{\text{decide}} + T_{\text{act}}
\label{eq:reaction_time}
\end{equation}

Для реактивных агентов $T_{\text{decide}} \ll T_{\text{planning}}$, что обеспечивает быстрое реагирование.

Проблема: реактивные агенты не учитывают долгосрочные последствия действий.

Решение: гибридная архитектура, сочетающая реактивность с планированием.

\begin{figure}[H]
\centering
\begin{tikzpicture}[node distance=1.2cm, auto, scale=0.75, transform shape, font=\small]
    \node[rectangle, draw, fill=blue!20, minimum width=1.2cm, minimum height=0.6cm] (sensor) {Sensor};
    \node[rectangle, draw, fill=green!20, right=of sensor, minimum width=1.2cm, minimum height=0.6cm] (processor) {Processor};
    \node[rectangle, draw, fill=yellow!20, below=of processor, minimum width=1.2cm, minimum height=0.6cm] (actuator) {Actuator};
    
    \draw[->, shorten >=2pt, shorten <=2pt] (sensor) -- (processor);
    \draw[->, shorten >=2pt, shorten <=2pt] (processor) -- (actuator);
    \draw[->, shorten >=2pt, shorten <=2pt] (actuator) -- (sensor);
\end{tikzpicture}
\caption{Архитектура реактивного агента \cite{wooldridge_agents,russell_artificial_intelligence}}
\label{fig:reactive_agent}
\end{figure}

\section{BDI-архитектура: решение проблемы целеполагания}

BDI-архитектура решает проблему целеполагания через формализацию убеждений, желаний и намерений \cite{bdi_architecture}. Проблема реактивных агентов — отсутствие явных целей.

В BDI-архитектуре агент оперирует тремя типами ментальных установок. Множество убеждений (Beliefs) определяется как:

\begin{equation}
B = \{b_1, b_2, \ldots, b_n\}, \quad b_i: \mathcal{S} \to [0, 1]
\label{eq:beliefs}
\end{equation}

где $b_i(s)$ — степень убеждённости в утверждении $i$ при состоянии $s$.

Множество желаний (Desires):

\begin{equation}
D = \{d_1, d_2, \ldots, d_m\}, \quad d_i: \mathcal{S} \to [0, 1]
\label{eq:desires}
\end{equation}

где $d_i(s)$ — желательность состояния $s$ для цели $i$.

Множество намерений (Intentions):

\begin{equation}
I = \{i_1, i_2, \ldots, i_k\}, \quad i_j: \mathcal{S} \times \mathcal{A} \to [0, 1]
\label{eq:intentions}
\end{equation}

где $i_j(s, a)$ — намерение выполнить действие $a$ в состоянии $s$ для достижения цели $j$.

Выбор намерения:

\begin{equation}
\text{Intention}^* = \arg\max_{i \in I} \left[ \text{Desire}(i) \cdot \text{Belief}(i) \cdot \text{Feasibility}(i) \right]
\label{eq:bdi_decision}
\end{equation}

где $\text{Desire}(i) = \sum_{d \in D} w_d \cdot d(s)$ — желательность намерения, $\text{Belief}(i) = \prod_{b \in B_i} b(s)$ — убеждённость в достижимости, $\text{Feasibility}(i) = P(\text{success}|i, s)$ — выполнимость намерения.

Байесовское обновление убеждений:

\begin{equation}
P(b|o) = \frac{P(o|b) P(b)}{P(o)} = \frac{P(o|b) P(b)}{\sum_{b' \in B} P(o|b') P(b')}
\label{eq:belief_update}
\end{equation}

где $o$ — новое наблюдение.

\section{Модель принятия решений: решение проблемы оптимизации}

Принятие решений в условиях неопределённости требует оптимизации ожидаемой полезности. Проблема заключается в учёте неопределённости и рисков.

Формализуем задачу принятия решений. Пусть пространство состояний системы определяется как:

\begin{equation}
\mathcal{S} = \{s_1, s_2, \ldots, s_n\}
\label{eq:state_space}
\end{equation}

Пространство действий агента:

\begin{equation}
\mathcal{A} = \{a_1, a_2, \ldots, a_m\}
\label{eq:action_space}
\end{equation}

Функция вознаграждения:

\begin{equation}
R: \mathcal{S} \times \mathcal{A} \times \mathcal{S} \to \mathbb{R}
\label{eq:reward_function}
\end{equation}

Функция полезности действия (ожидаемое вознаграждение):

\begin{equation}
U(s, a) = \mathbb{E}[R(s, a, s')] + \gamma \max_{a'} V(s')
\label{eq:utility_function}
\end{equation}

где $\gamma \in [0,1]$ — коэффициент дисконтирования, $V(s')$ — функция ценности состояния $s'$.

Функция ценности состояния (уравнение Беллмана) \cite{goodfellow_deep_learning,murphy_machine_learning}:

\begin{equation}
V(s) = \max_{a \in \mathcal{A}} \left[ R(s, a) + \gamma \sum_{s' \in \mathcal{S}} P(s'|s, a) V(s') \right]
\label{eq:bellman_equation}
\end{equation}

где $P(s'|s, a)$ — вероятность перехода в состояние $s'$ при выполнении действия $a$ в состоянии $s$.

Q-функция (функция качества действия):

\begin{equation}
Q(s, a) = R(s, a) + \gamma \sum_{s' \in \mathcal{S}} P(s'|s, a) \max_{a'} Q(s', a')
\label{eq:q_function}
\end{equation}

Оптимальная политика агента:

\begin{equation}
\pi^*(s) = \arg\max_{a \in \mathcal{A}} Q(s, a)
\label{eq:optimal_policy}
\end{equation}

Проблема неопределённости: вероятности переходов $P(s'|s, a)$ могут быть неизвестны или неточны.

Решение через учёт неопределённости:

\begin{equation}
\text{Action} = \arg\max_{a \in \mathcal{A}} \left[ U(s, a) - \lambda \cdot \text{Var}[R(s, a)] \right]
\label{eq:decision_making_uncertainty}
\end{equation}

где $\lambda$ — параметр неприятия риска, $\text{Var}[R(s, a)]$ — дисперсия вознаграждения.

Для нормального распределения вознаграждения:

\begin{equation}
\text{Var}[R(s, a)] = \mathbb{E}[(R(s, a) - \mathbb{E}[R(s, a)])^2]
\label{eq:reward_variance}
\end{equation}

\section{Rule-based vs ML-based: проблема выбора подхода}

Rule-based подходы прозрачны, но не адаптивны. ML-based подходы адаптивны, но менее интерпретируемы. Проблема заключается в выборе оптимального подхода для конкретной задачи.

В rule-based агенте используется набор правил: $\mathcal{R} = \{r_1, r_2, \ldots, r_k\}$.

Правило $r_i$ имеет вид: $\text{if } \text{condition}_i \text{ then } \text{action}_i$.

Применение правил:

\begin{equation}
\text{Action} = \text{action}_j, \quad j = \arg\min_{i: \text{condition}_i(s) = \text{true}} \text{priority}(r_i)
\label{eq:rule_application}
\end{equation}

где $\text{priority}(r_i)$ — приоритет правила $r_i$.

Проблема: правила статичны и не адаптируются к изменениям в системе.

В ML-based агенте обучение политики происходит на исторических данных следующим образом:

\begin{equation}
\pi^* = \arg\max_{\pi} \mathbb{E}_{\tau \sim \pi} \left[ \sum_{t=0}^{T} \gamma^t R(s_t, a_t) \right]
\label{eq:ml_policy}
\end{equation}

где $\tau = (s_0, a_0, s_1, a_1, \ldots)$ — траектория, $T$ — горизонт планирования.

Аппроксимация Q-функции нейронной сетью \cite{goodfellow_deep_learning,acm_machine_learning}:

\begin{equation}
Q(s, a; \theta) \approx Q^*(s, a)
\label{eq:q_approximation}
\end{equation}

где $\theta$ — параметры нейронной сети.

Обучение через минимизацию ошибки:

\begin{equation}
L(\theta) = \mathbb{E}[(Q(s, a; \theta) - (R(s, a) + \gamma \max_{a'} Q(s', a'; \theta')))^2]
\label{eq:q_learning_loss}
\end{equation}

где $\theta'$ — параметры целевой сети (target network).

Преимущество: адаптация к изменениям через переобучение.

Гибридный подход: комбинация правил и ML:

\begin{equation}
\text{Action} = \begin{cases}
\text{rule\_based}(s) & \text{если } \text{confidence}(\text{rule}) > \theta_{\text{rule}} \\
\text{ml\_based}(s) & \text{иначе}
\end{cases}
\label{eq:hybrid_approach}
\end{equation}

\section{Жизненный цикл агента: проблема непрерывного обучения}

Агент должен адаптироваться к изменениям в системе через непрерывное обучение. Проблема заключается в балансе между эксплуатацией (exploitation) и исследованием (exploration).

Жизненный цикл агента состоит из следующих этапов: $\{\text{Init}, \text{Observe}, \text{Analyze}, \text{Decide}, \text{Act}, \text{Learn}\}$.

На этапе Observe агент собирает данные:

\begin{equation}
D_t = \{d_1, d_2, \ldots, d_n\}, \quad d_i = (s_i, a_i, r_i, s_i')
\label{eq:experience_data}
\end{equation}

где $(s_i, a_i, r_i, s_i')$ — кортеж состояние-действие-вознаграждение-следующее состояние.

На этапе Analyze агент обнаруживает аномалии:

\begin{equation}
\text{Anomaly}(s_t) = \begin{cases}
\text{true} & \text{если } f_{\text{anomaly}}(s_t) > \theta \\
\text{false} & \text{иначе}
\end{cases}
\label{eq:anomaly_detection}
\end{equation}

где $f_{\text{anomaly}}$ — функция обнаружения аномалий, $\theta$ — порог.

На этапе Decide агент выбирает действие:

\begin{equation}
a_t = \begin{cases}
\arg\max_{a} Q(s_t, a) & \text{с вероятностью } 1 - \epsilon \\
\text{random}(a) & \text{с вероятностью } \epsilon
\end{cases}
\label{eq:epsilon_greedy}
\end{equation}

где $\epsilon$ — параметр exploration (исследования).

На этапе Learn агент обновляет модель:

\begin{equation}
\theta_{t+1} = \theta_t - \alpha \nabla_\theta L(\theta_t)
\label{eq:learning_update}
\end{equation}

где $\alpha$ — скорость обучения, $\nabla_\theta L$ — градиент функции потерь.

Баланс exploration-exploitation через $\epsilon$-greedy стратегию:

\begin{equation}
\epsilon_t = \epsilon_0 \cdot e^{-\lambda t}
\label{eq:epsilon_decay}
\end{equation}

где $\epsilon_0$ — начальное значение exploration, $\lambda$ — скорость затухания.

\section{LLM-based агенты: решение проблемы анализа неструктурированных данных и помощи разработчикам}

Традиционные методы машинного обучения требуют структурированных данных и явного определения признаков. Однако значительная часть информации в распределённых системах содержится в неструктурированных данных: текстовых логах, сообщениях об ошибках, описаниях инцидентов. Large Language Models (LLM) — большие языковые модели — способны понимать естественный язык и извлекать информацию из неструктурированных текстов, что делает их мощным инструментом для анализа логов и помощи разработчикам в диагностике проблем.

LLM представляет собой функцию, которая преобразует последовательность токенов в распределение вероятностей над следующим токеном:

\begin{equation}
P(w_{t+1} | w_1, w_2, \ldots, w_t) = \text{LLM}(w_1, w_2, \ldots, w_t)
\label{eq:llm_next_token}
\end{equation}

где $w_i$ — токены входной последовательности.

Для анализа логов LLM используется в режиме генерации ответов на вопросы (question answering). Пусть $L = \{l_1, l_2, \ldots, l_n\}$ — множество логов, $Q$ — вопрос о проблеме. LLM генерирует ответ:

\begin{equation}
A = \text{LLM}(Q, L, \text{Context})
\label{eq:llm_qa}
\end{equation}

где $\text{Context}$ — дополнительный контекст о системе.

Вероятность того, что ответ $A$ релевантен вопросу $Q$ при данных логах $L$:

\begin{equation}
P(A|Q, L) = \prod_{i=1}^{|A|} P(a_i | Q, L, a_1, \ldots, a_{i-1})
\label{eq:llm_relevance}
\end{equation}

где $a_i$ — токены ответа, $|A|$ — длина ответа.

Для извлечения структурированной информации из логов используется техника извлечения именованных сущностей (Named Entity Recognition, NER). LLM может идентифицировать сущности в логах:

\begin{equation}
\text{Entities}(L) = \{(e_i, t_i, c_i) : e_i \in E, t_i \in T, c_i \in [0, 1]\}
\label{eq:llm_ner}
\end{equation}

где $E$ — множество типов сущностей (сервисы, ошибки, метрики), $T$ — множество типов, $c_i$ — уверенность в извлечении.

LLM-агент анализирует логи, метрики и трейсы и генерирует рекомендации для разработчиков следующим образом:

\begin{equation}
\text{Recommendations} =
  \text{LLM-Agent}(
    \text{Logs},
    \text{Metrics},
    \text{Traces},
    \text{Problem Description}
  )
\label{eq:llm_recommendations}
\end{equation}

Рекомендации включают:
\begin{itemize}
    \item Описание проблемы на естественном языке
    \item Вероятные причины проблемы на основе анализа логов и метрик
    \item Предлагаемые действия для исправления с указанием конкретных файлов кода или конфигураций
    \item Ссылки на релевантные части кода или документации
    \item Оценку приоритета проблемы и срочности исправления
\end{itemize}

Качество рекомендаций оценивается через следующую метрику:

\begin{equation}
\text{Quality}(R) = \alpha \cdot \text{Accuracy}(R) + \beta \cdot \text{Relevance}(R) + \gamma \cdot \text{Actionability}(R) + \delta \cdot \text{Clarity}(R)
\label{eq:recommendation_quality}
\end{equation}

где $\text{Accuracy}$ — точность описания проблемы, $\text{Relevance}$ — релевантность рекомендаций, $\text{Actionability}$ — выполнимость рекомендаций, $\text{Clarity}$ — ясность объяснений, $\alpha, \beta, \gamma, \delta$ — веса компонентов.

Время генерации рекомендаций:

\begin{equation}
T_{\text{generation}} = T_{\text{preprocessing}} + T_{\text{llm\_inference}} + T_{\text{postprocessing}}
\label{eq:llm_generation_time}
\end{equation}

где $T_{\text{preprocessing}}$ — время подготовки данных для LLM (фильтрация релевантных логов, агрегация метрик), $T_{\text{llm\_inference}}$ — время инференса LLM, $T_{\text{postprocessing}}$ — время обработки ответа (валидация, форматирование, добавление ссылок).

Преимущество LLM-агентов заключается в их способности понимать контекст и генерировать объяснения на естественном языке, что делает их особенно полезными для помощи разработчикам, которые могут не быть экспертами в конкретной области проблемы. LLM может связать сообщение об ошибке в логах с конкретным изменением в коде, что помогает разработчикам быстрее найти и исправить проблему.

Гибридный подход: комбинация rule-based, ML-based и LLM-based методов:

\begin{equation}
\text{Decision} = \begin{cases}
\text{rule-based}(s) & \text{если } \text{confidence}(\text{rule}) > \theta_{\text{rule}} \\
\text{ml-based}(s) & \text{если } \text{confidence}(\text{ml}) > \theta_{\text{ml}} \\
\text{llm-based}(s) & \text{иначе}
\end{cases}
\label{eq:hybrid_decision}
\end{equation}

где $\theta_{\text{rule}}, \theta_{\text{ml}}$ — пороги уверенности для rule-based и ML-based методов.

Для объединения результатов различных методов используется ансамбль:

\begin{equation}
\text{Final Decision} = \sum_{i \in \{\text{rule}, \text{ml}, \text{llm}\}} w_i \cdot \text{Decision}_i
\label{eq:ensemble_decision}
\end{equation}

где $w_i$ — веса методов, определяемые на основе их исторической точности и времени генерации ответа.

Время принятия решения в гибридном подходе:

\begin{equation}
T_{\text{hybrid}} = \min(T_{\text{rule}}, T_{\text{ml}}) + P(\text{need\_llm}) \cdot T_{\text{llm}}
\label{eq:hybrid_time}
\end{equation}

где $P(\text{need\_llm})$ — вероятность того, что требуется LLM для анализа, $T_{\text{rule}}, T_{\text{ml}}, T_{\text{llm}}$ — времена генерации решений различными методами.

Это позволяет использовать быстрые методы для простых случаев и более мощные LLM для сложных случаев, требующих анализа неструктурированных данных.

Изложенный в главе аппарат интеллектуальных агентов описывает полный цикл автоматизированного принятия решений в распределённых системах — от реактивного реагирования на отдельные события до долгосрочного планирования с учётом неопределённости и риска. Формулировка задачи выбора действия как максимизации полезности \(U(s,a)\) и уравнения Беллмана для функции ценности \(V(s)\) задают общую рамку оптимизации поведения агента во времени.

Реактивные и BDI-архитектуры формализуют, соответственно, быстрый цикл «восприятие–решение–действие» и целеполагание через убеждения, желания и намерения, а модели rule-based и ML-based политики описывают два полюса по оси «интерпретируемость–адаптивность». Жизненный цикл агента с \(\epsilon\)-greedy стратегией и непрерывным обучением уточняет, каким образом знания о системе обновляются по мере накопления опыта эксплуатации.

Введение LLM-ориентированных компонентов дополняет классический спектр методов возможностью анализа неструктурированных данных и генерации контекстных рекомендаций, а гибридные схемы выбора действий и ансамбли решений позволяют сочетать rule-based, ML-based и LLM-based механизмы в единую систему. В дальнейшем эти архитектурные и алгоритмические элементы используются как строительные блоки при проектировании конкретной системы интеллектуальных агентов для диагностики и самовосстановления, описанной в последующих главах.
