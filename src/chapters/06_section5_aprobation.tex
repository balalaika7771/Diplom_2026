\chapter{Апробация}

\section{Условия апробации}

Апробация разработанной системы интеллектуальных автономных агентов проводилась в два этапа: внутреннее тестирование на инфраструктуре кафедры и пилотное внедрение в тестовую среду компании-партнёра.

\textbf{Этап 1: лабораторная апробация} (сентябрь--ноябрь 2025). Система развёрнута на кластере Kubernetes из 3 узлов (спецификация в разделе~3). Тестирование проводилось на 10 микросервисах e-commerce платформы с инжекцией отказов через Chaos Mesh. Длительность непрерывной работы --- 8 недель, за которые система обработала 312 инцидентов (147 синтетических, 165 из production-логов).

\textbf{Этап 2: пилотное внедрение} (декабрь 2025 -- февраль 2026). Система интегрирована в тестовый контур распределённой платформы, обслуживающей 47 микросервисов. Агенты работали в режиме shadow (параллельно с дежурными инженерами, без автоматического выполнения действий восстановления) в течение 12 недель.

\section{Интеграция с инструментами наблюдаемости}

В ходе апробации подтверждена совместимость системы со стеком наблюдаемости:

\begin{table}[H]
\centering
\caption{Результаты интеграционного тестирования}
\label{tab:integration_results}
\small
\begin{tabular}{|p{3.5cm}|p{3cm}|p{5cm}|}
\hline
\textbf{Инструмент} & \textbf{Версия} & \textbf{Результат интеграции} \\
\hline
Prometheus & 2.47 & Сбор 1200 метрик/с без деградации \\
\hline
Elasticsearch & 8.10 & Индексация логов с задержкой $< 2$ с \\
\hline
Jaeger & 1.50 & Анализ трейсов для 47 сервисов \\
\hline
Grafana & 10.2 & Дашборды состояния агентов \\
\hline
OpenTelemetry & 1.28 & Унифицированный сбор телеметрии \\
\hline
Kafka & 3.6 & Шина событий, throughput 5000 msg/s \\
\hline
\end{tabular}
\end{table}

Накладные расходы системы агентов на инфраструктуру: дополнительное потребление CPU --- 8\% от ёмкости кластера, RAM --- 4.2 ГБ на 3 агента, сетевой трафик --- менее 1\% от общего.

\section{Результаты лабораторной апробации}

За 8 недель лабораторного тестирования система продемонстрировала следующие результаты:

\begin{table}[H]
\centering
\caption{Результаты лабораторной апробации (312 инцидентов, 8 недель)}
\label{tab:lab_aprobation_results}
\small
\begin{tabular}{|p{5cm}|c|c|}
\hline
\textbf{Показатель} & \textbf{Значение} & \textbf{Целевое} \\
\hline
Обнаружено инцидентов & 298 из 312 (95.5\%) & $> 90\%$ \\
\hline
Корректно диагностирована причина & 271 из 298 (90.9\%) & $> 85\%$ \\
\hline
Успешное автовосстановление & 243 из 271 (89.7\%) & $> 80\%$ \\
\hline
Среднее время обнаружения & 2.3 мин & $< 5$ мин \\
\hline
Среднее время диагностики & 4.7 мин & $< 10$ мин \\
\hline
Ложные срабатывания & 24 (7.7\%) & $< 15\%$ \\
\hline
Некорректные действия Safe Executor & 0 & 0 \\
\hline
\end{tabular}
\end{table}

14 пропущенных инцидентов относились к двум категориям: медленные утечки ресурсов с периодом развития более 4 часов (9 случаев) и аномалии в нетипичных метриках, отсутствовавших в обучающей выборке (5 случаев). 28 случаев некорректной диагностики причины были связаны с каскадными отказами, затрагивающими более 5 сервисов одновременно.

\section{Результаты пилотного внедрения}

В режиме shadow-мониторинга система анализировала инциденты параллельно с дежурными инженерами. Сравнение проводилось по двум критериям: скорость определения корневой причины и корректность диагноза.

\begin{table}[H]
\centering
\caption{Сравнение: система агентов vs дежурные инженеры (shadow-режим, 12 недель)}
\label{tab:shadow_comparison}
\small
\begin{tabular}{|p{4.5cm}|c|c|c|}
\hline
\textbf{Метрика} & \textbf{Инженеры} & \textbf{Агенты} & \textbf{$\Delta$} \\
\hline
Среднее время диагностики, мин & $38.4 \pm 15.2$ & $6.1 \pm 2.8$ & $-84\%$ \\
\hline
Корректность диагноза, \% & 87.3 & 89.1 & $+1.8$ п.п. \\
\hline
Пропущенные инциденты, \% & 4.2 & 5.8 & $+1.6$ п.п. \\
\hline
Ложные тревоги в сутки & $1.2$ & $2.8$ & $+133\%$ \\
\hline
\end{tabular}
\end{table}

Система агентов значительно превосходит инженеров по скорости диагностики ($-84\%$) при сопоставимой корректности. Инженеры реже пропускают инциденты (4.2\% vs 5.8\%), но это преимущество нивелируется при комбинированном использовании: совместная работа агентов и инженеров снизила долю пропущенных инцидентов до 1.4\%.

Количество ложных тревог у агентов выше (2.8 vs 1.2 в сутки), что является направлением дальнейшей оптимизации. Анализ показал, что 60\% ложных тревог связаны с плановыми деплоями, которые система некорректно классифицировала как аномалии.

\section{Оценка безопасности}

За весь период апробации (20 недель) система Safe Executor:

\begin{itemize}
    \item Заблокировала 18 потенциально опасных действий, предложенных модулем восстановления (перезапуск критических сервисов при высокой нагрузке, масштабирование сверх лимитов ресурсов).
    \item Не допустила ни одного некорректного действия восстановления, прошедшего валидацию.
    \item Средняя задержка валидации: 1.2 с (при пороге $r_{\max} = 0.3$).
\end{itemize}

\section{Публикации и представление результатов}

Результаты исследования были представлены:

\begin{enumerate}
    \item Доклад на научном семинаре кафедры прикладной информатики и теории вероятностей РУДН (ноябрь 2025).
    \item Основные положения работы использованы при разработке прототипа системы и его апробации в тестовой среде компании-партнёра.
\end{enumerate}
