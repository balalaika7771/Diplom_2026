\chapter{Архитектура проектируемой системы}

\section{Введение: проектирование интегрированной системы диагностики и восстановления}

Системы мониторинга, диагностики и восстановления часто реализуются как слабо связанные подсистемы, что приводит к задержкам между обнаружением инцидента, его анализом и выполнением действий восстановления. На практике это выражается в повышенном MTTR: результаты мониторинга передаются операторам, а анализ логов, локализация причины и выбор действий выполняются вручную.

Проектируемая система ориентирована на автоматизацию контура «обнаружение–диагностика–решение–восстановление» и интегрирует модули обнаружения аномалий, диагностики, принятия решений и безопасного выполнения действий. Модули функционируют автономно, их взаимодействие организовано так, чтобы минимизировать время восстановления при соблюдении требований к точности и безопасности.

Основные требования к архитектуре:
\begin{itemize}
    \item Модульность — компоненты должны быть независимыми и заменяемыми
    \item Расширяемость — возможность добавления новых методов обнаружения и диагностики
    \item Безопасность — гарантии того, что действия восстановления не навредят системе
    \item Производительность — способность обрабатывать большие объёмы данных в реальном времени
    \item Надёжность — система должна продолжать работать даже при сбоях отдельных компонентов
\end{itemize}

В данной главе рассматриваются математические модели архитектуры системы, которые обеспечивают выполнение этих требований.

\section{Проблема интеграции модулей обнаружения, диагностики и восстановления}

Основная проблема заключается в отсутствии единой архитектуры, объединяющей все этапы процесса: от обнаружения аномалий до безопасного восстановления. Существующие системы работают изолированно, что создаёт задержки и снижает эффективность реагирования на инциденты.

Математическая формулировка задачи: необходимо построить систему, которая минимизирует время восстановления:

\begin{equation}
\text{MTTR} = T_{\text{detection}} + T_{\text{diagnosis}} + T_{\text{recovery}} \to \min
\label{eq:mttr_minimization}
\end{equation}

при ограничениях:

\begin{align}
P(\text{detection}) &> \theta_{\text{precision}} \label{eq:precision_constraint} \\
P(\text{false\_positive}) &< \theta_{\text{fpr}} \label{eq:fpr_constraint} \\
P(\text{safe\_recovery}) &> \theta_{\text{safety}} \label{eq:safety_constraint}
\end{align}

где $\theta_{\text{precision}}, \theta_{\text{fpr}}, \theta_{\text{safety}}$ — пороговые значения точности, ложных срабатываний и безопасности.

\section{Архитектурные принципы: решение проблемы модульности}

Принцип модульности обеспечивает независимость компонентов и возможность их замены. Проблема монолитных систем — сложность изменений и тестирования.

Для количественной оценки модульности системы рассмотрим её разбиение на модули: $M = \{M_1, M_2, \ldots, M_n\}$.

Связанность модулей (coupling):

\begin{equation}
C(M_i, M_j) = \frac{|\text{interfaces}(M_i, M_j)|}{|\text{interfaces}(M_i)| + |\text{interfaces}(M_j)|}
\label{eq:coupling}
\end{equation}

где $\text{interfaces}(M_i, M_j)$ — множество интерфейсов между модулями $M_i$ и $M_j$.

Связность модуля (cohesion):

\begin{equation}
H(M_i) = \frac{|\text{internal\_dependencies}(M_i)|}{|\text{all\_dependencies}(M_i)|}
\label{eq:cohesion}
\end{equation}

Принцип высокой связности и низкой связанности:

\begin{equation}
\forall i: H(M_i) \to \max, \quad \forall i \neq j: C(M_i, M_j) \to \min
\label{eq:modularity_principle}
\end{equation}

\section{Модули системы: решение проблемы специализации}

Каждый модуль решает специфическую задачу, что обеспечивает эффективность и возможность независимого масштабирования.

\subsection{Data Collector: проблема агрегации разнородных данных}

Различные источники данных (Prometheus, ELK, Jaeger) имеют различные форматы и протоколы \cite{prometheus_docs,elasticsearch_docs,jaeger_docs,opentelemetry}. Проблема заключается в унификации данных для последующего анализа.

Пусть входные данные из источника $i$ представляются как $D_i = \{d_{i,1}, d_{i,2}, \ldots\}$.

Функция нормализации:

\begin{equation}
D_{\text{normalized}} = \bigcup_{i=1}^{n} f_{\text{normalize},i}(D_i)
\label{eq:data_normalization}
\end{equation}

где $f_{\text{normalize},i}$ — функция нормализации для источника $i$.

Единая модель данных:

\begin{equation}
d = (\text{timestamp}, \text{source}, \text{type}, \text{value}, \text{metadata})
\label{eq:unified_data_model}
\end{equation}

где $\text{type} \in \{\text{metric}, \text{log}, \text{trace}\}$.

Время сбора данных:

\begin{equation}
T_{\text{collection}} = \max_{i=1}^{n} T_{\text{collect},i}
\label{eq:collection_time}
\end{equation}

где $T_{\text{collect},i}$ — время сбора данных из источника $i$.

\subsection{Anomaly Analyzer: проблема многомерного анализа}

Аномалии могут проявляться в различных измерениях данных. Проблема заключается в объединении результатов различных методов обнаружения.

Для объединения результатов различных методов рассмотрим результат метода $i$ как $r_i \in [0, 1]$ — оценку аномальности.

Взвешенное объединение результатов:

\begin{equation}
\text{Anomaly Score} = \sum_{i=1}^{m} w_i \cdot r_i, \quad \sum_{i=1}^{m} w_i = 1
\label{eq:ensemble_anomaly}
\end{equation}

где $w_i$ — вес метода $i$, определяется на основе его точности:

\begin{equation}
w_i = \frac{\exp(\alpha \cdot \text{accuracy}_i)}{\sum_{j=1}^{m} \exp(\alpha \cdot \text{accuracy}_j)}
\label{eq:method_weight}
\end{equation}

где $\alpha$ — параметр температуры.

Финальное решение:

\begin{equation}
\text{Anomaly} = \begin{cases}
\text{true} & \text{если } \text{Anomaly Score} > \theta \\
\text{false} & \text{иначе}
\end{cases}
\label{eq:anomaly_decision}
\end{equation}

\subsection{LLM Agent: проблема анализа неструктурированных данных и помощи разработчикам}

Традиционные модули работают со структурированными данными — метриками и трейсами. Однако значительная часть информации о проблемах содержится в неструктурированных текстовых логах. Модуль LLM Agent использует большие языковые модели для анализа логов, генерации объяснений проблем и помощи разработчикам в быстром исправлении проблем.

Модуль получает на вход логи $L$, контекст от Diagnostic Engine $C$ и генерирует рекомендации для разработчиков следующим образом:

\begin{equation}
\text{Recommendations} = \text{LLM-Agent}(L, C, \text{Metrics}, \text{Traces})
\label{eq:llm_agent_recommendations}
\end{equation}

Время генерации рекомендаций:

\begin{equation}
T_{\text{llm}} = T_{\text{log\_filtering}} + T_{\text{context\_preparation}} + T_{\text{llm\_inference}} + T_{\text{postprocessing}}
\label{eq:llm_agent_time}
\end{equation}

где $T_{\text{log\_filtering}}$ — время фильтрации релевантных логов, $T_{\text{context\_preparation}}$ — время подготовки контекста для LLM, $T_{\text{llm\_inference}}$ — время инференса LLM, $T_{\text{postprocessing}}$ — время обработки ответа.

Качество рекомендаций оценивается через метрику:

\begin{equation}
\text{Quality} = \alpha \cdot \text{Accuracy} + \beta \cdot \text{Time\_to\_Fix} + \gamma \cdot \text{Developer\_Satisfaction}
\label{eq:llm_quality_metric}
\end{equation}

где $\text{Time\_to\_Fix}$ — время, которое разработчик потратил на исправление проблемы с помощью рекомендаций, $\text{Developer\_Satisfaction}$ — оценка разработчика полезности рекомендаций.

Интеграция с другими модулями:

\begin{equation}
\text{Final Action} =
  \begin{cases}
    \text{Orchestrator Decision},
      & \text{если } \text{confidence}(\text{orchestrator}) > \theta, \\
    \text{LLM Recommendation},
      & \text{если } \text{confidence}(\text{llm}) > \theta \\
      & \quad \land \text{needs\_human\_review}, \\
    \text{Human Review},
      & \text{иначе}.
  \end{cases}
\label{eq:llm_integration}
\end{equation}

\subsection{Diagnostic Engine: проблема построения причинно-следственных графов}

Определение корневой причины инцидента требует анализа причинно-следственных связей между событиями. Проблема заключается в неполноте данных и неопределённости связей.

Причинно-следственный граф представляется как $G = (V, E)$, где $V$ — множество вершин (события, метрики, сервисы), $E$ — множество рёбер (причинно-следственные связи). Матрица смежности определяется следующим образом:

\begin{equation}
\mathbf{A}_{ij} = \begin{cases}
1 & \text{если } P(v_i \to v_j) > \theta \\
0 & \text{иначе}
\end{cases}
\label{eq:adjacency_matrix}
\end{equation}

Вероятность причинно-следственной связи:

\begin{equation}
P(v_i \to v_j | \mathbf{m}, \mathbf{l}, \mathbf{t}) = \sigma\left(\sum_{k} w_k \cdot f_k(v_i, v_j, \mathbf{m}, \mathbf{l}, \mathbf{t})\right)
\label{eq:causal_probability}
\end{equation}

где $f_k$ — функции признаков:
\begin{itemize}
    \item Временная близость: $f_{\text{time}}(v_i, v_j) = \exp(-|t_i - t_j|/\tau)$
    \item Корреляция: $f_{\text{corr}}(v_i, v_j) = |\rho(v_i, v_j)|$
    \item Зависимость в трейсе: $f_{\text{trace}}(v_i, v_j) = \mathbf{1}[\text{parent}(v_j) = v_i]$
\end{itemize}

Критерий причинности (Granger causality) \cite{lamport_distributed_systems,lynch_distributed_algorithms}:

\begin{equation}
v_j \text{ causes } v_i \Leftrightarrow \text{var}(\epsilon_{i|v_j}) < \text{var}(\epsilon_{i|\neg v_j})
\label{eq:granger_causality}
\end{equation}

где $\epsilon_{i|v_j}$ — ошибка предсказания $v_i$ с учётом $v_j$.

Построение графа:

\begin{equation}
G = \{(v_i, v_j) : P(v_i \to v_j) > \theta \land \text{Granger}(v_j, v_i)\}
\label{eq:causal_graph_building}
\end{equation}

Поиск корневой причины:

\begin{equation}
\text{RootCause} = \arg\min_{v \in V} \left[ \text{in-degree}(v) + \lambda \cdot (1 - P(\text{anomaly}|v)) \right]
\label{eq:root_cause}
\end{equation}

где $\lambda$ — параметр баланса между структурой графа и вероятностью аномалии.

\begin{figure}[H]
\centering
\begin{tikzpicture}[node distance=0.8cm, auto, scale=0.7, transform shape, font=\small]
    \node[rectangle, draw, fill=blue!20, minimum width=1cm, minimum height=0.6cm] (collector) {Collector};
    \node[rectangle, draw, fill=green!20, right=of collector, minimum width=1cm, minimum height=0.6cm] (analyzer) {Analyzer};
    \node[rectangle, draw, fill=yellow!20, right=of analyzer, minimum width=1cm, minimum height=0.6cm] (diagnostic) {Diagnostic};
    \node[rectangle, draw, fill=purple!20, above=of diagnostic, minimum width=1cm, minimum height=0.6cm] (llm_agent) {LLM};
    \node[rectangle, draw, fill=red!20, below=of analyzer, minimum width=1cm, minimum height=0.6cm] (orchestrator) {Orchestrator};
    \node[rectangle, draw, fill=orange!20, below=of orchestrator, minimum width=1cm, minimum height=0.6cm] (executor) {Executor};
    
    \draw[->, shorten >=2pt, shorten <=2pt] (collector) -- (analyzer);
    \draw[->, shorten >=2pt, shorten <=2pt] (analyzer) -- (diagnostic);
    \draw[->, shorten >=2pt, shorten <=2pt] (diagnostic) -- (orchestrator);
    \draw[->, shorten >=2pt, shorten <=2pt] (collector) -- (llm_agent);
    \draw[->, shorten >=2pt, shorten <=2pt] (diagnostic) -- (llm_agent);
    \draw[->, shorten >=2pt, shorten <=2pt] (llm_agent) -- (orchestrator);
    \draw[->, shorten >=2pt, shorten <=2pt] (orchestrator) -- (executor);
    \draw[->, dashed, shorten >=2pt, shorten <=2pt] (executor) -- (collector);
\end{tikzpicture}
\caption{Архитектура системы интеллектуальных агентов}
\label{fig:system_architecture}
\end{figure}

\section{Механизмы оркестрации: проблема координации множественных агентов}

При наличии нескольких агентов необходимо координировать их действия для избежания конфликтов \cite{wooldridge_agents,multi_agent_systems}. Проблема заключается в достижении консенсуса при различных мнениях агентов \cite{herlihy_art_multiprocessor}.

Рассмотрим множество агентов $\mathcal{A} = \{A_1, A_2, \ldots, A_n\}$.

Мнение агента $A_i$ о действии $a$:

\begin{equation}
o_i(a) \in [0, 1]
\label{eq:agent_opinion}
\end{equation}

Вес агента на основе его точности:

\begin{equation}
w_i = \frac{\exp(\alpha \cdot \text{accuracy}_i)}{\sum_{j=1}^{n} \exp(\alpha \cdot \text{accuracy}_j)}
\label{eq:agent_weight}
\end{equation}

где $\text{accuracy}_i$ — историческая точность агента $A_i$, $\alpha$ — параметр температуры.

Консенсусное решение:

\begin{equation}
\text{Decision} = \arg\max_{a \in \mathcal{A}} \sum_{i=1}^{n} w_i \cdot o_i(a)
\label{eq:consensus}
\end{equation}

Алгоритм консенсуса DeGroot:

\begin{equation}
o_i^{(t+1)}(a) = \sum_{j=1}^{n} T_{ij} \cdot o_j^{(t)}(a)
\label{eq:degroot_consensus}
\end{equation}

где $T_{ij}$ — элемент матрицы доверия, $\sum_{j} T_{ij} = 1$.

Условие сходимости к консенсусу:

\begin{equation}
\lim_{t \to \infty} o_i^{(t)}(a) = \bar{o}(a) = \frac{1}{n} \sum_{i=1}^{n} o_i^{(0)}(a)
\label{eq:consensus_convergence}
\end{equation}

Скорость сходимости определяется вторым по величине собственным значением матрицы $T$:

\begin{equation}
\lambda_2(T) < 1 \Rightarrow \text{convergence}
\label{eq:convergence_condition}
\end{equation}

Взвешенное голосование:

\begin{equation}
\text{Vote}(a) = \sum_{i=1}^{n} w_i \cdot \mathbf{1}[o_i(a) > \theta_{\text{support}}]
\label{eq:weighted_voting}
\end{equation}

где $\theta_{\text{support}}$ — порог поддержки действия.

Порог принятия решения:

\begin{equation}
\text{Decision} = \begin{cases}
a^* & \text{если } \text{Vote}(a^*) > \theta_{\text{consensus}} \cdot \sum_{i=1}^{n} w_i \\
\text{abstain} & \text{иначе}
\end{cases}
\label{eq:decision_threshold}
\end{equation}

где $\theta_{\text{consensus}} \in [0,1]$ — порог консенсуса.

\section{Безопасное исполнение решений: проблема минимизации рисков}

Автоматическое выполнение действий по восстановлению несёт риски ошибок. Проблема заключается в обеспечении безопасности при сохранении эффективности.

Вероятность успешного выполнения действия определяется как произведение вероятностей успеха на каждом этапе валидации:

\begin{equation}
P(\text{success}|a, s) = \prod_{i=1}^{k} P_i(\text{success}|a, s)
\label{eq:success_probability}
\end{equation}

где $P_i$ — вероятность успеха на этапе $i$ валидации, $k$ — количество этапов валидации.

Ожидаемая полезность с учётом риска:

\begin{equation}
EU(a, s) =
  P(\text{success} \mid a, s) \cdot U(\text{success}, a, s)
  + (1 - P(\text{success} \mid a, s)) \cdot U(\text{failure}, a, s)
\label{eq:expected_utility}
\end{equation}

где $U(\text{success}, a, s)$ — полезность успешного выполнения, $U(\text{failure}, a, s)$ — полезность (отрицательная) при неудаче.

Функция риска:

\begin{equation}
\text{Risk}(a, s) = P(\text{failure}|a, s) \cdot \text{Cost}(\text{failure}, a, s)
\label{eq:risk_function}
\end{equation}

Критерий безопасности:

\begin{equation}
\text{Safe}(a, s) = \begin{cases}
\text{true} & \text{если } \text{Risk}(a, s) < \theta_{\text{risk}} \land P(\text{success}|a, s) > \theta_{\text{success}} \\
\text{false} & \text{иначе}
\end{cases}
\label{eq:safety_criterion}
\end{equation}

где $\theta_{\text{risk}}$ — максимально допустимый риск, $\theta_{\text{success}}$ — минимально допустимая вероятность успеха.

Постепенное применение (gradual rollout):

\begin{equation}
\text{Apply}(a, s, \alpha) = \begin{cases}
\text{partial} & \text{если } \alpha < 1 \land \text{Safe}(a, s) \\
\text{full} & \text{если } \alpha = 1 \land \text{Safe}(a, s) \land \text{Verify}(\alpha) \\
\text{abort} & \text{иначе}
\end{cases}
\label{eq:gradual_application}
\end{equation}

где $\alpha \in [0,1]$ — доля применения действия, $\text{Verify}(\alpha)$ — функция верификации на доле $\alpha$.

Модель отката (rollback):

При обнаружении ошибки система откатывается к предыдущему состоянию:

\begin{equation}
\text{Rollback}(s_t, s_{t-k}) = \arg\min_{a \in \mathcal{A}_{\text{rollback}}} \left[ \|s_t - s_{t-k}\| + \lambda \cdot \text{Cost}(a) \right]
\label{eq:rollback_model}
\end{equation}

где $s_{t-k}$ — состояние $k$ шагов назад, $\lambda$ — параметр стоимости отката, $\|\cdot\|$ — метрика расстояния между состояниями.

Вероятность успешного отката:

\begin{equation}
P(\text{rollback\_success}) = P(\text{state\_saved}) \cdot P(\text{rollback\_executable})
\label{eq:rollback_success}
\end{equation}

Ограничение изменений для предотвращения каскадных эффектов:

\begin{equation}
\sum_{i=1}^{n} |\Delta m_i| \leq \Delta_{\max}
\label{eq:change_limit}
\end{equation}

где $\Delta m_i$ — изменение метрики $m_i$, $\Delta_{\max}$ — максимально допустимое суммарное изменение.

Уровни безопасности определяются через функцию риска:

\begin{equation}
\text{SafetyLevel}(a) = \begin{cases}
\text{Low} & \text{если } \text{Risk}(a) < \theta_1 \\
\text{Medium} & \text{если } \theta_1 \leq \text{Risk}(a) < \theta_2 \\
\text{High} & \text{если } \theta_2 \leq \text{Risk}(a) < \theta_3 \\
\text{Critical} & \text{если } \text{Risk}(a) \geq \theta_3
\end{cases}
\label{eq:safety_levels}
\end{equation}

где $\theta_1 < \theta_2 < \theta_3$ — пороговые значения риска.

Изложенная архитектура системы интеллектуальных агентов задаёт формальный каркас интеграции модулей сбора данных, анализа аномалий, причинно-следственной диагностики, оркестрации решений и безопасного исполнения. Оптимизационная постановка задачи минимизации MTTR при ограничениях на точность, частоту ложных срабатываний и уровень риска Recovery-действий определяет критерии качества для всей цепочки обработки инцидентов.

Принципы модульности, унификации данных и ансамблевого анализа аномалий обеспечивают устойчивость архитектуры к расширению и замене отдельных методов, а причинно-следственный граф и механизмы консенсуса между агентами формализуют путь от наблюдаемых симптомов к выбору конкретных действий. Встроенные критерии безопасности и поддержка LLM-агента для работы с неструктурированными логами обеспечивают как строгость автоматических решений, так и возможность их верификации и уточнения человеком.

В дальнейшем, при реализации прототипа и проведении экспериментов, данная архитектура служит инвариантной основой: отдельные алгоритмы обнаружения, диагностики и принятия решений могут варьироваться, но структура взаимодействия модулей и используемые количественные критерии остаются неизменными.
