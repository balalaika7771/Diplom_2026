\chapter{Введение}

\section{Актуальность темы исследования}

Современные распределённые вычислительные системы стали основой цифровой инфраструктуры, обеспечивая работу критически важных сервисов в различных областях: финансовые транзакции, электронная коммерция, облачные вычисления, интернет вещей. Рост сложности и масштаба таких систем приводит к увеличению частоты инцидентов и сложности их диагностики \cite{google_sre_book}.

Статистика показывает, что среднее время восстановления (MTTR) после инцидента в крупных распределённых системах составляет от 30 минут до нескольких часов \cite{netflix_chaos}. При этом до 70\% времени восстановления тратится на диагностику проблемы, а не на её устранение \cite{pagerduty_incident_response}. Это приводит к значительным экономическим потерям: час простоя критического сервиса может стоить компании от десятков тысяч до миллионов долларов \cite{amazon_outage_cost}.

Традиционные подходы к мониторингу, основанные на метриках и пороговых значениях, не справляются с растущей сложностью современных систем. Они генерируют большое количество ложных срабатываний, не учитывают взаимосвязи между компонентами системы и не способны предсказывать деградацию до возникновения критических проблем \cite{observability_engineering}.

Концепция наблюдаемости (observability) предлагает новый подход к пониманию состояния системы через анализ метрик, логов и трейсов. Однако современные инструменты наблюдаемости требуют значительных усилий от инженеров для интерпретации данных и принятия решений \cite{charity_majors_observability}.

\begin{figure}[H]
\centering
\begin{tikzpicture}[scale=0.8]
\begin{axis}[
    xlabel={Год},
    ylabel={Количество инцидентов (тыс.)},
    title={Рост количества инцидентов в распределённых системах},
    grid=major,
    legend pos=north west,
]
\addplot[blue,thick] coordinates {
    (2015, 50) (2016, 65) (2017, 85) (2018, 110) (2019, 140) (2020, 180) (2021, 220) (2022, 270) (2023, 320)
};
\addplot[red,thick,dashed] coordinates {
    (2015, 30) (2016, 35) (2017, 40) (2018, 45) (2019, 50) (2020, 55) (2021, 60) (2022, 65) (2023, 70)
};
\legend{Общее количество инцидентов, Количество критических инцидентов}
\end{axis}
\end{tikzpicture}
\caption{Рост количества инцидентов в распределённых системах за последние годы \cite{aiops_survey,pagerduty_incident_response}}
\label{fig:incidents_growth}
\end{figure}

На рисунке \ref{fig:incidents_growth} показан рост количества инцидентов в распределённых системах, что подтверждает актуальность проблемы автоматизации диагностики и восстановления.

\begin{table}[H]
\centering
\caption{Статистика времени восстановления после инцидентов}
\label{tab:recovery_time_stats}
\small
\begin{tabular}{|p{3.5cm}|c|c|c|}
\hline
\textbf{Тип инцидента} & \textbf{MTTR, мин} & \textbf{Диагностика, \%} & \textbf{Восстановление, \%} \\
\hline
Отказ сервиса & 45 & 65 & 35 \\
\hline
Деградация производительности & 120 & 70 & 30 \\
\hline
Утечка данных & 180 & 55 & 45 \\
\hline
Каскадный отказ & 240 & 60 & 40 \\
\hline
\end{tabular}
\end{table}

В таблице \ref{tab:recovery_time_stats} представлена статистика времени восстановления, показывающая, что большая часть времени тратится на диагностику проблемы.

Развитие технологий машинного обучения и больших языковых моделей открывает новые возможности для автоматизации диагностики и восстановления систем. Интеллектуальные автономные агенты могут анализировать большие объёмы данных наблюдаемости, выявлять паттерны аномалий, строить причинно-следственные связи и принимать решения о восстановлении \cite{aiops_survey}.

\section{Проблема исследования}

Существует противоречие между:
\begin{itemize}
    \item Ростом сложности распределённых систем и необходимостью быстрого обнаружения и устранения проблем
    \item Ограниченными возможностями традиционных систем мониторинга и потребностью в интеллектуальной диагностике
    \item Большими объёмами данных наблюдаемости и необходимостью их эффективного анализа для принятия решений
    \item Требованиями к безопасности операций восстановления и необходимостью автоматизации процессов
\end{itemize}

\section{Цель исследования}

Разработать архитектуру и методы построения интеллектуальных автономных агентов для диагностики, мониторинга и самовосстановления распределённых вычислительных систем на основе комплексного анализа данных наблюдаемости.

\section{Объект исследования}

Объектом исследования являются распределённые вычислительные системы и процессы их диагностики, мониторинга и восстановления.

\section{Предмет исследования}

Предметом исследования являются методы и алгоритмы построения интеллектуальных автономных агентов для автоматической диагностики и самовосстановления распределённых систем на основе анализа данных наблюдаемости.

\section{Гипотеза исследования}

Гипотеза исследования формулируется следующим образом. Использование интеллектуальных автономных агентов, объединяющих методы машинного обучения для обнаружения аномалий, причинно-следственный анализ инцидентов и безопасные механизмы самовосстановления, приводит к статистически значимому улучшению показателей надёжности и эксплуатационной эффективности распределённых вычислительных систем по сравнению с существующими подходами.

Формально, пусть \(MTTR\) обозначает среднее время восстановления после инцидента, а \(R\) — интегральную метрику надёжности системы (например, вероятность безотказной работы на заданном интервале времени). Тогда гипотеза утверждает существование конфигурации разработанных агентов и процедур их интеграции в эксплуатационную среду, для которой выполняются неравенства
\[
  MTTR_{\text{agent}} < MTTR_{\text{baseline}}, \quad
  R_{\text{agent}} > R_{\text{baseline}},
\]
\label{eq:hypothesis_ineq}
где индекс \(\text{baseline}\) соответствует существующей системе без интеллектуальных агентов, а \(\text{agent}\) — системе с внедрёнными автономными агентами диагностики и самовосстановления.

\section{Задачи исследования}

Для достижения поставленной цели необходимо решить следующие задачи:

\begin{enumerate}
    \item Провести анализ теоретических основ распределённых вычислительных систем, их архитектур и принципов построения устойчивых систем.
    
    \item Исследовать концепцию наблюдаемости как фундамента диагностики распределённых систем, проанализировать современные подходы к сбору и анализу метрик, логов и трейсов.
    
    \item Провести сравнительный анализ backend-подходов к построению устойчивых систем: синхронных и асинхронных архитектур, REST и gRPC, event-driven систем, CQRS, принципов идемпотентности и паттернов устойчивости.
    
    \item Изучить инженерные парадигмы DevOps и SRE, методы автоматизации развёртывания, стратегии самовосстановления в Kubernetes, ограничения современных систем мониторинга.
    
    \item Исследовать методы обнаружения аномалий и прогнозирования деградации: статистические методы, методы машинного обучения для временных рядов, применение LLM для анализа логов.
    
    \item Проанализировать архитектуры интеллектуальных автономных агентов: реактивные архитектуры, BDI-архитектуры, rule-based vs ML-based vs LLM-based подходы, жизненный цикл агента.
    
    \item Разработать архитектуру интеллектуального автономного агента для диагностики и самовосстановления, определить требования, архитектурные принципы, модули системы, механизмы оркестрации и безопасного исполнения решений.
    
    \item Провести сравнительный анализ инженерных парадигм: монолит vs микросервисы, push vs pull модели, event vs request-driven архитектуры, react vs actor модели, rule-based vs ML-based диагностика.
    
    \item Реализовать прототип системы интеллектуальных автономных агентов, разработать структуру модулей, схемы взаимодействия, параметры конфигурации, моделирование сценариев самовосстановления.
    
    \item Провести экспериментальное исследование эффективности разработанного подхода: разработать методику экспериментов, определить сценарии тестирования, метрики оценки (MTTR, latency, precision, recall), провести анализ результатов.
    
    \item Оценить практическую значимость разработанного решения: определить области применения в индустрии, описать возможности интеграции в реальные системы.
\end{enumerate}

\section{Методы исследования}

В работе используются следующие методы исследования:

\begin{enumerate}
    \item \textbf{Теоретический анализ} — изучение научной литературы, стандартов и документации по распределённым системам, наблюдаемости, машинному обучению и архитектурам агентов.
    
    \item \textbf{Сравнительный анализ} — сопоставление различных подходов к построению устойчивых систем, методов обнаружения аномалий, архитектур агентов.
    
    \item \textbf{Математическое моделирование} — разработка моделей для анализа временных рядов метрик, построения причинно-следственных графов, принятия решений агентами.
    
    \item \textbf{Проектирование архитектуры} — разработка архитектуры интеллектуальных автономных агентов с использованием принципов модульности, расширяемости и безопасности.
    
    \item \textbf{Экспериментальное моделирование} — создание прототипа системы и проведение экспериментов на синтетических и реальных данных.
    
    \item \textbf{Статистический анализ} — оценка эффективности разработанного подхода через метрики точности, полноты, времени восстановления.
\end{enumerate}

\section{Основные обозначения и термины}

В работе используются следующие ключевые обозначения и метрики:

\begin{itemize}
    \item \(MTTR\) (Mean Time To Recovery) — среднее время восстановления системы после инцидента, мин.
    \item \(MTBF\) (Mean Time Between Failures) — среднее время между отказами, мин/ч.
    \item \(A\) (Availability) — доступность системы, вероятность работоспособности в произвольный момент времени; в главах 2 и 11 определяется через \(MTBF\) и \(MTTR\) (см. формулы~(\ref{eq:availability_formula}), (\ref{eq:availability})).
    \item \(R(t)\) — функция надёжности, вероятность безотказной работы системы или компонента до момента времени \(t\).
    \item \(\text{Precision}\), \(\text{Recall}\), \(F_1\) — метрики качества обнаружения аномалий, определённые через матрицу ошибок (см. главу 11).
    \item \(\text{SLO}\), \(\text{SLI}\), \(\text{Error Budget}\) — показатели инженерии надёжности (SRE), связывающие целевые и фактические уровни качества обслуживания (глава 5).
    \item \(\text{Anomaly}(t)\) — индикатор аномального состояния в момент времени \(t\), принимающий значение 1 при обнаружении аномалии и 0 в противном случае.
    \item \(\text{Degradation}(t)\) — индикатор прогнозируемой деградации системы в будущем интервале (глава 6).
\end{itemize}

Все прочие обозначения (например, вектора состояний \(x(t)\), \(\mathbf{m}(t)\), вероятностные функции и операторы сдвига) вводятся локально в соответствующих главах при первом использовании.

\section{Научная новизна}

Научная новизна работы заключается в следующем:

\begin{enumerate}
    \item Разработана комплексная архитектура интеллектуальных автономных агентов, объединяющая методы машинного обучения для обнаружения аномалий, причинно-следственный анализ инцидентов и безопасные механизмы самовосстановления.
    
    \item Предложен новый подход к построению причинно-следственных графов инцидентов на основе комплексного анализа метрик, логов и трейсов, позволяющий повысить точность диагностики корневых причин проблем.
    
    \item Разработана методология интеграции больших языковых моделей для анализа логов и автоматической генерации гипотез о причинах инцидентов.
    
    \item Предложен механизм безопасного самовосстановления с использованием fail-safe принципов и валидации действий агентов перед их выполнением.
    
    \item Разработана система метрик для оценки эффективности интеллектуальных автономных агентов, учитывающая не только точность обнаружения аномалий, но и безопасность и эффективность действий по восстановлению.
\end{enumerate}

\section{Практическая значимость}

Практическая значимость работы определяется следующими аспектами:

\begin{enumerate}
    \item Разработанные методы и архитектура могут быть использованы при проектировании систем мониторинга и автоматизации операций в крупных IT-компаниях и облачных провайдерах.
    
    \item Результаты работы способствуют снижению операционных затрат за счёт автоматизации диагностики и восстановления, снижения времени простоя систем.
    
    \item Предложенные подходы могут быть интегрированы в существующие платформы наблюдаемости (Prometheus, Grafana, OpenTelemetry) для расширения их функциональности \cite{prometheus_docs,grafana_docs,opentelemetry_docs}.
    
    \item Разработанные методы могут быть использованы при внедрении практик SRE (Site Reliability Engineering) в организациях \cite{google_sre_book,sre_principles}.
    
    \item Результаты работы могут быть применены при проектировании систем для критически важных приложений: финансовых систем, систем здравоохранения, промышленной автоматизации \cite{iso_25010,amazon_outage_cost}.
\end{enumerate}

\begin{table}[H]
\centering
\caption{Ожидаемые улучшения метрик при использовании интеллектуальных агентов}
\label{tab:expected_improvements}
\small
\begin{tabular}{|p{3.5cm}|c|c|c|}
\hline
\textbf{Метрика} & \textbf{Текущее} & \textbf{Ожидаемое} & \textbf{Улучшение} \\
\hline
MTTR, мин & 120 & 66 & -45\% \\
\hline
Точность обнаружения, \% & 78 & 94 & +16\% \\
\hline
Ложные срабатывания, \% & 35 & 12 & -23\% \\
\hline
Время диагностики, мин & 78 & 30 & -62\% \\
\hline
\end{tabular}
\end{table}

В таблице \ref{tab:expected_improvements} представлены ожидаемые улучшения ключевых метрик при использовании разработанных интеллектуальных агентов.

Для каждой метрики
\[
  m \in \{
    \text{MTTR},
    \text{precision},
    \text{false\_positive\_rate},
    \text{diagnosis\_time}
  \}
\]
относительное улучшение определяется как
\begin{equation}
  I_m = \frac{m_{\text{baseline}} - m_{\text{agent}}}{m_{\text{baseline}}} \cdot 100\%,
\end{equation}
где \(m_{\text{baseline}}\) — значение метрики в существующей системе, \(m_{\text{agent}}\) — значение метрики при использовании интеллектуальных агентов. Для метрик, интерпретируемых по принципу «чем больше, тем лучше» (например, точность обнаружения аномалий), используется модифицированное определение:
\begin{equation}
  I_m = \frac{m_{\text{agent}} - m_{\text{baseline}}}{m_{\text{baseline}}} \cdot 100\%.
\end{equation}
Такая формализация позволяет однозначно сопоставлять эффект внедрения разработанного подхода в разных сценариях эксплуатации и использовать полученные значения в последующем статистическом анализе.

\section{Структура работы}

Диссертация состоит из введения, 13 глав, заключения, списка использованных источников и приложений.

\textbf{Глава 1} (Введение) определяет актуальность темы, проблему, цель, задачи, методы исследования, научную новизну и практическую значимость работы.

\textbf{Глава 2} посвящена теоретическим основам распределённых вычислительных систем: архитектурам, теореме CAP, PACELC, микросервисным архитектурам, паттернам коммуникации.

\textbf{Глава 3} рассматривает наблюдаемость как фундамент диагностики: метрики, логи, трейсы, причинно-следственные графы, ограничения современных стеков.

\textbf{Глава 4} анализирует backend-подходы к устойчивым системам: синхронные и асинхронные архитектуры, REST/gRPC/event-driven, CQRS, идемпотентность, паттерны устойчивости.

\textbf{Глава 5} изучает DevOps, SRE и архитектуры обеспечения стабильности: GitOps, IaC, Kubernetes self-healing, стратегии развёртывания, автоматизацию тестирования.

\textbf{Глава 6} исследует методы обнаружения аномалий и прогнозирования деградации: статистические методы, ML для временных рядов, LLM-анализ логов.

\textbf{Глава 7} анализирует архитектуры интеллектуальных автономных агентов: реактивные архитектуры, BDI, rule-based vs ML-based vs LLM-based подходы.

\textbf{Глава 8} представляет архитектуру проектируемой системы: требования, принципы, модули, оркестрацию, безопасное исполнение.

\textbf{Глава 9} проводит сравнительный анализ инженерных парадигм: монолит vs микросервисы, push vs pull, event vs request-driven, react vs actor модели.

\textbf{Глава 10} описывает реализацию прототипа: структуру модулей, схемы, параметры, моделирование сценариев самовосстановления.

\textbf{Глава 11} представляет экспериментальное исследование: методику, сценарии, метрики, результаты экспериментов.

\textbf{Глава 12} оценивает практическую значимость: применение в индустрии, экономическую оценку, интеграцию в реальные системы.

\textbf{Глава 13} содержит заключение с основными выводами, результатами и перспективами развития.