% Термины и определения (по усмотрению исполнителя НИР)
\chapter*{ТЕРМИНЫ И ОПРЕДЕЛЕНИЯ}
\addcontentsline{toc}{chapter}{ТЕРМИНЫ И ОПРЕДЕЛЕНИЯ}

В настоящем отчете о НИР применяют следующие термины с соответствующими определениями:

\begin{description}
    \item[Наблюдаемость (Observability)] — свойство системы, позволяющее понять её внутреннее состояние на основе анализа внешних выходных данных (метрик, логов, трейсов), без необходимости модификации кода или добавления специальных точек инструментации.
    
    \item[Распределённая вычислительная система] — система, состоящая из множества независимых вычислительных узлов, взаимодействующих через сеть для достижения общей цели, при этом узлы могут быть географически распределены.
    
    \item[Автономный агент] — программный компонент, способный самостоятельно принимать решения и выполнять действия в заданной среде на основе анализа доступной информации, без прямого вмешательства человека.
    
    \item[Интеллектуальный агент] — автономный агент, использующий методы искусственного интеллекта и машинного обучения для анализа данных, принятия решений и адаптации к изменяющимся условиям.
    
    \item[Диагностика аномалий (Anomaly Detection)] — процесс автоматического обнаружения отклонений в поведении системы от нормального состояния, указывающих на потенциальные проблемы или инциденты.
    
    \item[Самовосстановление (Self-Healing)] — способность системы автоматически обнаруживать, диагностировать и устранять проблемы без вмешательства оператора, обеспечивая непрерывность работы.
    
    \item[Метрика (Metric)] — количественная мера состояния системы или процесса в определённый момент времени, представленная в виде числового значения.
    
    \item[Лог (Log)] — структурированная или неструктурированная запись событий, происходящих в системе, с указанием времени и контекста.
    
    \item[Трейс (Trace)] — запись пути выполнения запроса через распределённую систему, включающая информацию о всех сервисах, через которые прошёл запрос, и времени выполнения на каждом этапе.
    
    \item[Причинно-следственный граф (Causal Graph)] — граф, отражающий причинно-следственные связи между событиями и компонентами системы, позволяющий определить корневую причину инцидента.
    
    \item[Среднее время до восстановления (MTTR — Mean Time To Recovery)] — среднее время, необходимое для восстановления системы после возникновения инцидента до полного восстановления работоспособности.
    
    \item[Среднее время между отказами (MTBF — Mean Time Between Failures)] — среднее время между последовательными отказами системы.
    
    \item[Service Level Objective (SLO)] — целевой уровень качества обслуживания, определяемый как метрика надёжности системы (например, доступность 99.9\%).
    
    \item[Service Level Indicator (SLI)] — измеримый показатель качества обслуживания, используемый для оценки соответствия SLO.
    
    \item[Service Level Agreement (SLA)] — соглашение об уровне обслуживания между поставщиком и потребителем услуги, определяющее обязательства по качеству.
    
    \item[Инцидент (Incident)] — событие, приводящее к нарушению нормальной работы системы или снижению качества обслуживания.
    
    \item[Аномалия (Anomaly)] — отклонение в поведении системы от ожидаемого нормального состояния, которое может указывать на проблему.
    
    \item[Деградация (Degradation)] — постепенное ухудшение характеристик системы, не приводящее к полному отказу, но снижающее качество обслуживания.
    
    \item[Fail-safe] — принцип проектирования систем, при котором система при обнаружении ошибки переходит в безопасное состояние, предотвращая дальнейшее ухудшение ситуации.
    
    \item[Circuit Breaker] — паттерн проектирования, предотвращающий каскадные отказы путём временного прекращения запросов к неработающему сервису.
    
    \item[Retry] — механизм повторной попытки выполнения операции при возникновении временной ошибки.
    
    \item[Idempotency] — свойство операции, при котором многократное выполнение операции даёт тот же результат, что и однократное выполнение.
    
    \item[Event-Driven Architecture] — архитектурный паттерн, при котором компоненты системы взаимодействуют через асинхронную передачу событий.
    
    \item[CQRS (Command Query Responsibility Segregation)] — паттерн разделения операций чтения и записи данных на отдельные модели и хранилища.
    
    \item[Микросервисная архитектура] — архитектурный стиль, при котором приложение разбивается на множество небольших независимых сервисов, взаимодействующих через API.
    
    \item[Orchestrator] — компонент системы, координирующий выполнение операций между несколькими сервисами или агентами.
    
    \item[GitOps] — методология управления инфраструктурой и конфигурацией через Git-репозиторий, где изменения в коде автоматически применяются к системе.
    
    \item[Infrastructure as Code (IaC)] — подход к управлению инфраструктурой через код, позволяющий версионировать и автоматизировать развёртывание.
    
    \item[Kubernetes] — платформа для оркестрации контейнеризированных приложений, обеспечивающая автоматическое масштабирование, самовосстановление и управление жизненным циклом контейнеров.
    
    \item[Prometheus] — система мониторинга и сбора метрик с временными рядами, использующая pull-модель для сбора данных.
    
    \item[Grafana] — платформа для визуализации и анализа метрик, логов и трейсов.
    
    \item[OpenTelemetry] — стандарт и набор инструментов для сбора телеметрии (метрик, логов, трейсов) из приложений.
    
    \item[Distributed Tracing] — технология отслеживания пути выполнения запроса через распределённую систему для диагностики проблем производительности и надёжности.
    
    \item[Feature Engineering] — процесс создания признаков из исходных данных для улучшения качества моделей машинного обучения.
    
    \item[Time Series Analysis] — анализ данных, представляющих последовательность значений, измеренных в последовательные моменты времени.
    
    \item[Precision] — метрика качества модели, показывающая долю правильно обнаруженных аномалий среди всех обнаруженных.
    
    \item[Recall] — метрика качества модели, показывающая долю правильно обнаруженных аномалий среди всех реальных аномалий.
    
    \item[F1-Score] — гармоническое среднее между precision и recall, используемое для оценки баланса между точностью и полнотой обнаружения.
    
    \item[False Positive Rate] — доля ложных срабатываний, когда система ошибочно определяет нормальное состояние как аномалию.
    
    \item[Latency] — задержка выполнения операции, измеряемая как время от момента отправки запроса до получения ответа.
    
    \item[Throughput] — пропускная способность системы, измеряемая как количество операций, выполняемых за единицу времени.
    
    \item[Availability] — доступность системы, измеряемая как доля времени, в течение которого система работает и доступна для использования.
    
    \item[Resilience] — способность системы восстанавливаться после сбоев и продолжать функционирование в условиях неблагоприятных воздействий.
    
    \item[Chaos Engineering] — практика проведения контролируемых экспериментов над распределённой системой для выявления слабых мест и повышения устойчивости.
\end{description}

\newpage
