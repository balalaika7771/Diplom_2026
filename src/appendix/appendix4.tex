\chapter{Примеры логов и сценарии инцидентов}

\section{Примеры структурированных логов}

\subsection{Нормальное состояние}

\begin{lstlisting}[caption={Пример структурированного лога: нормальное состояние}, label=lst:log_normal]
{
  "timestamp": "2024-01-15T10:30:00Z",
  "level": "INFO",
  "service": "order-service",
  "message": "Order processed successfully",
  "order_id": "12345",
  "duration_ms": 45,
  "trace_id": "abc123def456"
}
\end{lstlisting}

\subsection{Аномальное состояние}

\begin{lstlisting}[caption={Пример структурированного лога: аномальное состояние}, label=lst:log_error]
{
  "timestamp": "2024-01-15T10:35:00Z",
  "level": "ERROR",
  "service": "order-service",
  "message": "Failed to connect to payment service",
  "error": "Connection timeout after 5s",
  "retry_count": 3,
  "trace_id": "abc123def456"
}
\end{lstlisting}

\section{Сценарии инцидентов}

\subsection{Сценарий 1: Высокая загрузка CPU}

\textbf{Симптомы:}
\begin{itemize}
    \item CPU usage > 90\% в течение 5 минут
    \item Увеличение latency запросов
    \item Увеличение ошибок 5xx
\end{itemize}

\textbf{Диагностика:}
\begin{itemize}
    \item Анализ метрик показывает недостаточное количество реплик
    \item Логи показывают увеличение времени обработки запросов
\end{itemize}

\textbf{Действие восстановления:}
\begin{itemize}
    \item Автоматическое масштабирование сервиса (увеличение реплик с 3 до 5)
    \item Мониторинг снижения нагрузки
\end{itemize}

\subsection{Сценарий 2: Каскадный отказ}

\textbf{Симптомы:}
\begin{itemize}
    \item Отказ основного сервиса
    \item Перегрузка зависимых сервисов
    \item Распространение ошибок по системе
\end{itemize}

\textbf{Диагностика:}
\begin{itemize}
    \item Построение причинно-следственного графа показывает цепочку зависимостей
    \item Определение корневой причины — отказ базы данных
\end{itemize}

\textbf{Действие восстановления:}
\begin{itemize}
    \item Включение circuit breaker для изоляции проблемного сервиса
    \item Переключение на резервную базу данных
    \item Постепенное восстановление соединений
\end{itemize}

\subsection{Сценарий 3: Прогнозируемая деградация производительности}

\textbf{Симптомы:}
\begin{itemize}
    \item Устойчивый рост latency ключевого эндпоинта (\texttt{/checkout}) без резкого скачка ошибок
    \item Прогноз LSTM показывает превышение порога SLA в ближайшие 10 минут
\end{itemize}

\textbf{Пример метрик:}
\begin{itemize}
    \item Текущее значение \(\text{Latency}_{\text{checkout}}(t) = 280\) мс при пороге SLA 300 мс
    \item Прогноз \(\hat{\text{Latency}}_{\text{checkout}}(t+5\ \text{мин}) = 340\) мс
\end{itemize}

\textbf{Диагностика:}
\begin{itemize}
    \item AnomalyDetector фиксирует предупреждение о деградации на основе прогноза
    \item Diagnostic Engine проверяет загрузку зависимых сервисов (\texttt{inventory}, \texttt{payment})
    \item Определяется, что узким местом является \texttt{payment-service} с высокой загрузкой CPU
\end{itemize}

\textbf{Действие восстановления:}
\begin{itemize}
    \item Превентивное масштабирование \texttt{payment-service} (увеличение числа реплик до расчётного значения)
    \item Перераспределение трафика с более равномерной балансировкой запросов
    \item Контроль фактического значения latency и откат изменений при отсутствии эффекта
\end{itemize}

\subsection{Сценарий 4: Неудачный релиз и откат конфигурации}

\textbf{Симптомы:}
\begin{itemize}
    \item После выката новой версии \texttt{payment-service} фиксируется рост доли ошибок 5xx
    \item Увеличение MTTR для инцидентов, связанных с платёжными операциями
\end{itemize}

\textbf{Пример логов:}
\begin{lstlisting}[caption={Фрагмент логов после неудачного релиза}, label=lst:log_failed_release]
{
  "timestamp": "2024-01-15T11:05:00Z",
  "level": "ERROR",
  "service": "payment-service",
  "message": "NullPointerException in PaymentProcessor",
  "version": "2.3.0",
  "trace_id": "xyz789"
}
\end{lstlisting}

\textbf{Диагностика:}
\begin{itemize}
    \item Diagnostic Engine связывает рост ошибок с конкретной версией сервиса (\texttt{version=2.3.0})
    \item Анализ метрик показывает ухудшение показателей только после момента релиза
    \item Safe Executor оценивает риск продолжения работы текущей версии как превышающий допустимый порог
\end{itemize}

\textbf{Действие восстановления:}
\begin{itemize}
    \item Откат \texttt{payment-service} к предыдущей стабильной версии (\texttt{2.2.5}) через Kubernetes Deployment rollback
    \item Верификация снижения \(\text{ErrorRate}\) и восстановления MTTR до базового уровня
    \item Запись результата сценария в журнал инцидентов для последующего анализа и доработки процесса релизов
\end{itemize}

