% Перечень условных обозначений, сокращений, терминов по ГОСТ 7.32-2017
\chapter*{ПЕРЕЧЕНЬ СОКРАЩЕНИЙ И ОБОЗНАЧЕНИЙ}
\addcontentsline{toc}{chapter}{ПЕРЕЧЕНЬ СОКРАЩЕНИЙ И ОБОЗНАЧЕНИЙ}

\vspace{1cm}

\begin{description}
    \item[API] Application Programming Interface — интерфейс программирования приложений
    \item[BDI] Belief-Desire-Intention — архитектура агентов на основе убеждений, желаний и намерений
    \item[CAP] Consistency, Availability, Partition tolerance — теорема о невозможности одновременного обеспечения консистентности, доступности и устойчивости к разделению
    \item[CI/CD] Continuous Integration / Continuous Deployment — непрерывная интеграция и развёртывание
    \item[CQRS] Command Query Responsibility Segregation — разделение ответственности команд и запросов
    \item[CPU] Central Processing Unit — центральный процессор
    \item[CRUD] Create, Read, Update, Delete — базовые операции с данными
    \item[DB] Database — база данных
    \item[DevOps] Development and Operations — методология разработки и эксплуатации
    \item[ETL] Extract, Transform, Load — извлечение, преобразование и загрузка данных
    \item[F1-Score] F1-мера — метрика качества классификации
    \item[GPU] Graphics Processing Unit — графический процессор
    \item[HTTP] HyperText Transfer Protocol — протокол передачи гипертекста
    \item[HTTPS] HyperText Transfer Protocol Secure — защищённый протокол передачи гипертекста
    \item[IaC] Infrastructure as Code — инфраструктура как код
    \item[ID] Identifier — идентификатор
    \item[I/O] Input/Output — ввод/вывод
    \item[IP] Internet Protocol — интернет-протокол
    \item[JSON] JavaScript Object Notation — формат обмена данными
    \item[K8s] Kubernetes — платформа оркестрации контейнеров
    \item[LLM] Large Language Model — большая языковая модель
    \item[ML] Machine Learning — машинное обучение
    \item[MTBF] Mean Time Between Failures — среднее время между отказами
    \item[MTTR] Mean Time To Recovery — среднее время до восстановления
    \item[NN] Neural Network — нейронная сеть
    \item[OS] Operating System — операционная система
    \item[PACELC] Partition tolerance, Availability, Consistency, Else Latency, Consistency — расширение теоремы CAP
    \item[PaaS] Platform as a Service — платформа как услуга
    \item[PID] Proportional-Integral-Derivative — пропорционально-интегрально-дифференциальный регулятор
    \item[QPS] Queries Per Second — запросов в секунду
    \item[RA] Reactive Architecture — реактивная архитектура
    \item[REST] Representational State Transfer — архитектурный стиль веб-сервисов
    \item[RFC] Request for Comments — запрос комментариев (стандарты интернета)
    \item[RPC] Remote Procedure Call — удалённый вызов процедур
    \item[SLA] Service Level Agreement — соглашение об уровне обслуживания
    \item[SLI] Service Level Indicator — индикатор уровня обслуживания
    \item[SLO] Service Level Objective — целевой уровень обслуживания
    \item[SRE] Site Reliability Engineering — инженерная надёжность сайтов
    \item[SQL] Structured Query Language — язык структурированных запросов
    \item[TCP] Transmission Control Protocol — протокол управления передачей
    \item[UDP] User Datagram Protocol — протокол пользовательских дейтаграмм
    \item[UI] User Interface — пользовательский интерфейс
    \item[URL] Uniform Resource Locator — унифицированный указатель ресурса
    \item[VM] Virtual Machine — виртуальная машина
    \item[YAML] YAML Ain't Markup Language — язык сериализации данных
\end{description}

\newpage
